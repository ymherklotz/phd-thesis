\chapter{Background}%
\label{sec:background}

\begin{quote}\itshape
  This chapter describes high-level synthesis and the current state-of-the-art
  optimisations used by \gls{HLS} tools, focusing in particular on static
  scheduling.  Next, common testing and verification workflows for \gls{HLS} are
  also described.  Finally, an overview of \compcert{} is given, on which
  \vericert{} is built.
\end{quote}

\section{Field Programmable Gate Arrays}%
\label{sec:bg:fpga}

This section briefly introduces \glspl{FPGA}, which is assumed to be the
final target for the hardware produced by Vericert, as well as the \gls{HLS}
tools that Vericert is directly compared against.

\Glspl{FPGA} are programmable hardware chips that can be used to implement and
run custom hardware without having to tape-out an actual integrated circuit.
\glspl{FPGA} comprise four main components:

\begin{description}
\item[\gls{LUT}] A \gls{LUT} can implement any kind of logic with a set number
  of inputs and a single output.  On an \gls{FPGA}, \glspl{LUT} are often
  grouped into larger programmable logic units called \emph{slices} that can
  handle multiple inputs and outputs.
\item[Programmable interconnect] The \glspl{LUT} are connected using
  programmable interconnects, so that these arbitrary logical units can also be
  connected in arbitrary ways, making it possible to implement any kind of
  hardware design.
\item[\gls{BRAM}] Instead of relying on implementing memories to store a large
  amount of data using \glspl{LUT}, there is often \gls{BRAM} on the \gls{FPGA},
  which provides efficient storage for a large amount of data.
\item[\gls{DSP} blocks] Finally, \glspl{FPGA} also often contain \gls{DSP}
  blocks, which can be used to implement common arithmetic functions
  efficiently, that may otherwise take up a lot of space if implemented using
  \glspl{LUT}.  Some common arithmetic functions that are often implemented
  using \gls{DSP} blocks include integer multipliers and multiply-accumulate
  operations.
\end{description}

\section{High-level Synthesis}%
\label{sec:bg:hls}

\Gls{HLS} is the transformation of software directly into hardware.  There are
many different types of \gls{HLS}, which can vary in terms of the languages they
accept or the devices that are targeted, however, they often share similar steps
in how the translation is performed, as they all go from a higher-level,
behavioural description of the algorithm to a timed hardware description.  In
this dissertation, we will assume that we are targeting \glspl{FPGA} instead of
\glspl{ASIC}, which lead to different resources that are available to the
\gls{HLS} tool to target.

The main steps performed in the translation of an \gls{HLS} tool is the
following~\cite{coussy09_introd_to_high_level_synth,canis13_legup}:

\begin{description}
\item[Compilation of specification] First, the input specification has to be
  compiled into a representation that can be used to apply all the necessary
  optimisations.  The input specification for most traditional \gls{HLS} tools
  is a restricted version of C or C++, however, \gls{HLS} tools such as Google
  XLS~\cite{google23_xls} use a \gls{DSL} based on communicating sequential
  processes~\cite{hoare78_commun_sequen_proces} as primary input specification.
  This specification is then turned into some intermediate language such as the
  LLVM \gls{IR}~\cite{lattner04_llvm}, MLIR~\cite{lattner21_mlir} or a custom
  representation of the code.

\item[Hardware resource allocation] Depending on if the hardware target is a
  specific \gls{FPGA} or an \gls{ASIC} working with a specific technology
  library, the \gls{HLS} tool will have to allocated resources differently.  For
  example, on \glspl{FPGA} there are only a limited number of \glspl{LUT},
  \gls{DSP} blocks available. However, tools often assume that an unlimited
  amount of resources are available instead.  This resource sharing can be too
  expensive for adders, as the logic for sharing the adder would have
  approximately the same cost as adding a separate adder.  However, multipliers
  and divider blocks should be shared, especially as divider blocks are often
  implemented in logic and are therefore quite expensive.

\item[Operation scheduling] The operations then need to be scheduled into a
  specific clock cycle, as the input specification language is normally an
  untimed behavioural representation.  This is where the spatial parallelism of
  the hardware can be taken advantage of, meaning operations that are not
  dependent on each other can run in parallel.  There are various possible
  scheduling methods that could be used, such as static or dynamic scheduling,
  which are described further in \cref{sec:bg:scheduling}.

\item[Operation and variable binding] After the operations have been scheduled,
  the operations and the variables need to be bound to hardware units and
  registers respectively.  It is often not that practical to share many hardware
  units though, as this requires adding multiplexers, which are often the same
  size as the hardware units themselves.  It is therefore more practical to only
  share operations that take up a lot of space, such as modulo or divide
  circuits, as well as multipliers if they are not available any more on the
  FPGA.

\item[Hardware description generation] Finally, the hardware description is
  generated from the code that was described in the intermediate language.  This
  is often a direct translation of the instructions into equivalent hardware
  constructs.
\end{description}

There are many examples of existing high-level synthesis tools, the most popular
ones being LegUp~\cite{canis13_legup}, Vivado
HLS~\cite{xilinx20_vivad_high_synth}, Catapult
C~\cite{mentor20_catap_high_level_synth} and Intel's OpenCL
SDK~\cite{intel20_sdk_openc_applic}.  These HLS tools all accept general
programming languages such as C or OpenCL, however, some HLS tools take in
languages that were designed for hardware such as
Handel-C~\cite{aubury96_handel_c_languag_refer_guide}, where time is encoded as
part of the semantics.

\subsection{Intermediate Language}%
\label{sec:bg:intermediate-language}

This section describes some possible characteristics of a language that could be
used as an input to an \gls{HLS} tool.  These languages normally require some
structure to allow for easier optimisation and analysis passes.  In particular,
it is often useful to have contiguous blocks of instructions that do not contain
any control-flow in one list.  This means that these instructions can safely be
rearranged by only looking at local information of the block itself, and in
particular it allows for complete removal of control-flow as only the data-flow
is important in that block.

\subsubsection{Basic blocks}

%TODO: Finish the basic blocks section

Basic blocks are the simplest form of structure, as these are only formed of
lists of instructions that do not include control-flow.

\subsubsection{Superblocks}

Superblocks extend the notion of basic blocks to contiguous regions without any
incoming control-flow, however, there can be multiple exits out of the block.
The main benefit of this definition is that due to the extra flexibility of
allowing multiple exits, the basic blocks can be extended, which often improves
most optimisations that make use of basic blocks.  However, the downside is that
the representation of the blocks can be more complex due to the introduction of
the extra control-flow.  Any analysis passes will have to take into account the
possibility of control-flow being present and many simplifications will not be
possible anymore.

\subsubsection{Hyperblocks}

Hyperblocks are also an extension of basic blocks similar to superblocks, but
instead of introducing special control-flow instructions into the block, every
instruction is predicated.  This leads to possibly more complex control-flow
than in both of the previous cases, however, it can be reasoned with using a
\gls{SAT} or \gls{SMT} solver.

\section{Scheduling}%
\label{sec:bg:scheduling}

\subsection{Static Scheduling}%
\label{sec:bg:static-scheduling}

Static scheduling is used by the majority of synthesis
tools~\cite{canis13_legup, xilinx20_vivad_high_synth,
  mentor20_catap_high_level_synth, intel20_sdk_openc_applic} and means that the
time at which each operation will execute is known at compile time.  Static
analysis is used to gather all the data dependencies between all the operations
to determine in which clock cycle the operations should be placed, or if these
operations can be parallelised.  Once the data-flow analysis has been performed,
various scheduling schemes can be taken into account to place operations in
various clock cycles.  Some common static-scheduling schemes are the following:

\begin{description}
\item[As soon as possible (ASAP)] scheduling will place operations into
  the first possible clock cycle that they can be executed.

\item[As late as possible (ALAP)] scheduling places operations into the last
  possible clock cycle, right before they are used by later operations.

\item[List scheduling] uses priorities associated with each operation and
  chooses operations from each priority level to be placed into the current
  clock cycle if all the data-dependencies are met.  This is done until all the
  resources for the current clock cycle are used up.
\end{description}

Static scheduling can normally produce extremely small circuits, however, it is
often not possible to guarantee that the circuits will have the best
throughput~\cite{cheng20_combin_dynam_static_sched_high_level_synth}, as this
requires extensive control-flow analysis and complex optimisations.  Especially
for loops, finding the optimal \gls{II} can be tricky if there are loads and
stores in the loop or if the loops are nested.

\subsection{Dynamic Scheduling}%
\label{sec:bg:dynamic-scheduling}

On the other hand, Dynamic
scheduling~\cite{josipovic18_dynam_sched_high_level_synth} does not require the
schedule to be known at compile time and instead it generates circuits using
tokens to schedule the operations in parallel at run time.  Whenever the data
for an operation is available, it sends a token to the next operation,
signalling that the data is ready to be read.  The next operation does not start
until all the required inputs to the operation are available, and once that is
the case, it computes the result and then sends another token declaring that the
result of that operation is also ready.  The benefit of this approach is that
only basic data-flow analysis is needed to connect the tokens correctly,
however, the scheduling is then done dynamically at run time, depending on how
long each primitive takes to finish and when the tokens activate the next
operations.

The benefits of this approach over static scheduling is that the latency of
these circuits is normally significantly lower than the latency of static
scheduled circuits, because they can take advantage of runtime information of
the circuit.  However, because of the signalling required to perform the runtime
scheduling, the area of these circuits is usually much larger than the area of
static scheduled circuits.  In addition to that, much more analysis is needed to
properly parallelise loads and stores to prevent bugs, which requires the
addition of buffers in certain locations.

An example of a dynamically scheduled synthesis tool is
Dynamatic~\cite{josipović18_dynam_sched_high_synth}, which uses a
\gls{LSQ}~\cite{josipovic17_out_of_order_load_store} to order memory operations
correctly even when loops are pipelined and there are dependencies between
iterations.  In addition to that, performance of the dynamically scheduled code
is improved by careful buffer
placement~\cite{josipovic20_buffer_placem_sizin_high_perfor_dataf_circuit},
which allows for better parallisation and pipelining of loops.

\section{Verification}%
\label{sec:bg:verification}

There are different kinds of verification tools, which can mostly be placed into
two categories: automatic theorem provers described in
\cref{sec:bg:automatic-theorem-provers} and interactive theorem provers
described in \cref{sec:bg:interactive-theorem-provers}.

\subsection{Automatic theorem provers}%
\label{sec:bg:automatic-theorem-provers}

Automatic theorem provers such as \gls{SAT} or \gls{SMT} solvers can be
characterised as decision procedures that will answer, for example, if a formula
is satisfiable or not.  However, by default these tools will only give the
answer to the initial query, without showing the reasoning.  The reasoning is
often quite complex as the \gls{SAT} or \gls{SMT} tool will implement many
optimisations to improve the performance of the decision procedure.

The main advantage of using an automatic theorem prover is that if one is
working in its constrained decidable theory, then it will be efficient at
proving or disproving if a formula is a theorem.  However, if the theorem
requires inductive arguments to prove, then the theorem prover might need some
manual help from the user by adding the right lemmas to its collection of facts
which the automatic procedure can use.  The proof itself though will still be
automatic, which means that many of the tedious cases in the proofs can be
ignored.

However, this is also the main disadvantage of automatic theorem provers,
because they do not provide details about the proof itself and often cannot
communicate why they cannot prove a theorem.  This means that as a user one has
to guess what theorems the prover is missing and try and add these to the fact
database.

Finally, automatic theorem provers do not provide reasoning for their final
answer by default, meaning one cannot check if the result is actually correct.
However, some SMT solvers support the generation of proof witnesses, which can
then be checked and reused in other theorem provers.  Some examples of these are
veriT~\cite{bouton09}, and these can then be integrated into Coq using
SMTCoq~\cite{armand11_modul_integ_sat_smt_solver}.

\subsection{Interactive theorem provers}%
\label{sec:bg:interactive-theorem-provers}

Interactive theorem provers, on the other hand, focus on checking proofs that
are provided to them.  These can either be written manually by the user, or
automatically generated by some decision procedure.  However, these two ways of
generating proofs can be combined, so the general proof structure can be
manually written, and simpler steps inside of the proof can be automatically
solved.

The main benefit of using an interactive theorem prover is that the proof is
there and can be checked by a small, trusted kernel.  This kernel does not need
to be heavily optimised, and can therefore be reasoned about.

The main cost of using an interactive theorem prover is the time it takes to
prove theorems, and the amount of formalisation that is needed to make the
proofs pass.  For a proof to be completed, one has to remove any axioms from the
proof, meaning even the smallest detail must be proven to continue.

\section{Unmechanised Verification of HLS}%
\label{sec:bg:unmechanised-verification-of-hls}

Work is being done to prove the equivalence between the generated hardware and
the original behavioural description in C.  An example of a tool that implements
this is Mentor's Catapult~\cite{mentor20_catap_high_level_synth}, which tries to
match the states in the register transfer level (RTL) description to states in
the original C code after an unverified translation.  This technique is called
translation validation~\cite{pnueli98_trans}, whereby the translation that the
HLS tool performed is proven to have been correct for that input, by showing
that they behave in the same way for all possible inputs.  Using translation
validation is quite effective for proving complex optimisations such as
scheduling~\cite{kim04_autom_fsmd, karfa06_formal_verif_method_sched_high_synth,
  chouksey20_verif_sched_condit_behav_high_level_synth} or code
motion~\cite{banerjee14_verif_code_motion_techn_using_value_propag,
  chouksey19_trans_valid_code_motion_trans_invol_loops}, however, the validation
has to be run every time the high-level synthesis is performed.  In addition to
that, the proofs are often not mechanised or directly related to the actual
implementation, meaning the verifying algorithm might be wrong and could give
false positives or false negatives.

More examples of translation validation for proofs about HLS
algorithms~\cite{karfa06_formal_verif_method_sched_high_synth,
  karfa07_hand_verif_high_synth, kundu07_autom,
  karfa08_equiv_check_method_sched_verif, kundu08_valid_high_level_synth,
  karfa10_verif_datap_contr_gener_phase, karfa12_formal_verif_code_motion_techn,
  chouksey19_trans_valid_code_motion_trans_invol_loops,
  chouksey20_verif_sched_condit_behav_high_level_synth} are performed using a
HLS tool called SPARK~\cite{gupta03_spark}.  These translation validation
algorithms can check the correctness of complicated optimisations such as code
motion or loop inversions.  However, even though the correctness of the verifier
is proven in the papers, the proof does translate directly to the algorithm that
was implemented for the verifier.  It is therefore possible that output is
accepted even though it is not equivalent to the input.  In addition to that,
these papers reason about the correctness of the algorithms in terms of the
intermediate language of SPARK, and does not extend to the high-level input
language that SPARK takes in, or the hardware description language that SPARK
outputs.

Finally there has also been work proving HLS correct without using translation
validation, but by directly showing that the translation is correct.  The first
instance of this is proving the BEDROC~\cite{chapman92_verif_bedroc} HLS tool is
correct.  This HLS tool converts a high-level description of an algorithm,
supporting loops and conditional statements, to a netlist and proves that the
output is correct.  It works in two stages, first generating a DFG from the
input language, HardwarePal.  It then optimises the DFG to improve the routing
of the design and also improving the scheduling of operations.  Finally, the
netlist is generated from the DFG by placing all the operations that do not
depend on each other into the same clock cycle.  Datapath and register
allocation is performed by an unverified clique partitioning algorithm.  The
equivalence proof between the DFG and HardwarePal is done by proof by
simulation, where it is proven that, given a valid input configuration, that
applying a translation or optimisation rule will result in a valid DFG with the
same behaviour as the input.

There has also been work on proving the translation from Occam to
gates~\cite{page91_compil_occam} correct using algebraic
proofs~\cite{jifeng93_towar}.  This translation resembles dynamic scheduling as
tokens are used to start the next operations.  However, Occam is timed by
design, and will execute assignments sequentially.  To take advantage of the
parallel nature of hardware, Occam uses explicit parallel constructs with
channels to share state.  Handel-C, a version of Occam with many more features
such as memory and pointers, has also been used to prove HLS correct, and is
described further in \cref{sec:bg:mechaniced-handel-c-to-netlist-translation}.

\section{Mechanised Compiler Proofs}%
\label{sec:bg:mechanised-compiler-proofs}

Even though a proof for the correctness of an algorithm might exist, this does
not guarantee that the algorithm itself behaves in the same way as the assumed
algorithm in the proof.  The implementation of the algorithm is separate from
the actual implementation, meaning there could be various implementation bugs in
the algorithm that cause it to behave incorrectly.  C compilers are a good
example of this, where many optimisations performed by the compilers have been
proven correct, however these proofs are not linked directly to the actual
implementations of these algorithms in GCC or Clang.  Yang et
al.~\cite{yang11_findin_under_bugs_c_compil} found more than 300 bugs in GCC and
Clang, many of them appearing in the optimisation phases of the compiler.  One
way to link the proofs to the actual implementations in these compilers is to
write the compiler in a language which allows for a theorem prover to check
properties about the algorithms.  This allows for the proofs to be directly
linked to the algorithms, ensuring that the actual implementations are proven
correct.  Yang et al.~\cite{yang11_findin_under_bugs_c_compil} found that
CompCert, a formally verified C Compiler, only had five bugs in all the
unverified parts of the compiler, meaning this method of proving algorithms
correct ensures a correct compiler.

\subsection{HLS Formalised in Isabelle}

Martin Ellis' work on correct synthesis~\cite{ellis08_correc} is the first
example of a mechanically verified high-level synthesis tool, also called
hardware/software compilers, as they produce software as well as hardware for
special functional units that should be hardware accelerated.  The main goal of
the thesis is to provide a framework to prove hardware/software compilers
correct, by defining semantics for an intermediate language (IR) which supports
partitioning of code into hardware and software parts, as well as a custom
netlist format which is used to describe the hardware parts and is the final
target of the hardware/software compiler.  The proof of correctness then says
that the hardware/software design should have the same behaviour as if the
design had only been implemented in software.  The framework used to prove the
correctness of the compilation from the IR to the netlist format is written in
Isabelle, which is a theorem prover comparable to Coq.  Any proofs in the
framework are therefore automatically checked by Isabelle for correctness.

This work first defines a static single assignment (SSA) IR, with the
capabilities of defining sections that should be hardware accelerated, and
sections that should be executed on the CPU.  This language supports
\emph{hyperblocks}, which are a list of basic blocks with one entry node and
multiple exit nodes, which is well suited for various HLS optimisations and
better scheduling.  However, as this is quite a low-level intermediate language,
the first thing that would be required is to create a tool that can actually
translate a high-level language to this intermediate language.  In addition to
that, they also describe a netlist format in which the hardware accelerated code
will be represented in.

As both the input IR and output netlist format have been designed from scratch,
they are not very useful for real world applications, as they require a
different back end to be implemented from existing compilers.  In addition to
that, it would most likely mean that the translation from higher-level language
to the IR is unverified and could therefore contain bugs.  As the resulting
netlist also uses a custom format, it cannot be fed directly to tools that can
then translate it to a bitstream to place onto an FPGA. The reason for designing
a custom IR and netlist was so that these were compatible with each other,
making proofs of equivalence between the two simpler.

Finally, it is unclear whether or not a translation algorithm from the IR to the
netlist format was actually implemented, as the only example in the thesis seems
to be compiled by hand to explain the proof.  There are also no benchmarks on
real input programs showing the efficiency of the translation algorithm, and it
is therefore unclear whether the framework would be able to prove more
complicated optimisations that a compiler might perform on the source code.  The
thesis seems to describe the correctness proofs by assuming a compiler exists
which outputs various properties that are needed by the equivalence proof, such
a mapping between variables and netlist wires and registers.

Even though this Isabelle framework does provide some equivalence relation
between an IR and a netlist and describes how the translation would be proven
correct by matching states to one another, it is actually not useable in
practice.  First of all, the IR is only represented in Isabelle and does not
have a parser or a compiler which can target this IR.  In addition to that, the
netlist format cannot be passed to any existing tool, to then be placed onto an
FPGA, meaning an additional, unverified translation would have to take place.
This further reduces the effectiveness of the correctness proof, as there are
various stages that are not addressed by the proofs and therefore have to be
assumed to be correct.

\subsection{Mechanised Handel-C to Netlist Translation}%
\label{sec:bg:mechaniced-handel-c-to-netlist-translation}

Handel-C~\cite{bowen98_handel_c_languag_refer_manual} is a C-like language for
hardware development.  It supports many C features such as assignments,
if-statements, loops, pointers and functions.  In addition to these constructs,
Handel-C also supports explicit timing constructs, such as explicit parallelism
as well as sequential or parallel assignment, similar to blocking and
nonblocking assignments in Verilog.  Perna et
al.~\cite{perna12_mechan_wire_wise_verif_handel_c_synth} developed a
mechanically verified Handel-C to netlist translation written in HOL. The
translation is based on previous work describing translation from Occam to gates
by Page et al.~\cite{page91_compil_occam}, which was proven correct by Jifeng et
al.~\cite{jifeng93_towar} using algebraic proofs.  As Handel-C is mostly an
extension of Occam with C-like operations, the translation from Handel-C to
gates can proceed in a similar way.

Perna et al. verify the compilation of a subset of Handel-C to gates, which does
not include memory, arrays or function calls.  In addition to the constructs
presented in Page et al., the prioritised choice construct is also added to the
Handel-C subset that is supported.  The verification proceeds by first defining
the algorithm to perform the compilation, chaining operations together with
start and end signals that determine the next construct which will be executed.
The circuits themselves are treated as black boxes by the compilation algorithm
and are chosen based on the current statement in Handel-C which is being
translated.  The compilation algorithm correctly has to link each black box to
one another using the start and end signals that the circuits expose to
correctly compile the Handel-C code.

The verification of the compilation is done by proving that the control signal
is propagated through the circuits correctly, and that each circuit is activated
at the right time.  It is also proven that the internal circuits of the
constructs also propagate the control signal in the correct way, and will be
active at the right clock cycle.  However, it is not proven that the circuits
have the same functionality as the Handel-C constructs, only that the control
signals are propagated in the correct manner.  In addition to that, the
assumption is made that the semantics for time are the same in Handel-C as well
as in the netlist format that is generated, which could be proven if the
constructs are shown to behave in the exact same way as the handel-C constructs.

\section{CompCert}%
\label{sec:bg:compcert}

\gls{CompCert}~\cite{leroy09_formal_verif_compil_back_end} is a formally
verified C compiler written in
Coq~\cite{bertot04_inter_theor_provin_progr_devel}.  Coq is a theorem prover,
meaning algorithms written in Coq can be reasoned about in Coq itself by proving
various properties about the algorithms.  To then run these algorithms that have
been proven correct, they can be extracted directly to OCaml code and then
executed, as there is a straightforward correspondence between Coq and OCaml
code.  During this translation, all the proofs are erased, as they are not
needed during the execution of the compiler, as they are only needed when the
correctness of the compiler needs to be checked.  With this process, one can
have a Compiler that satisfies various correctness properties and can therefore
be proven to preserve the behaviour of the code.

CompCert contains eleven intermediate languages, which are used to gradually
translate C code into assembly that has the same behaviour.  Proving the
translation directly without going through the intermediate languages would be
infeasible, especially with the many optimisations that are performed during the
translation, as assembly is very different to the abstract C code it receives.
The first three intermediate languages (C\#minor, C\#minorgen, Cminor) are used
to transform Clight into a more assembly like language called register transfer
language (RTL).  This language consist of a control-flow graph of instructions,
and is therefore well suited for various compiler optimisations such as constant
propagation, dead-code elimination or function
inlining~\cite{tristan08_formal_verif_trans_valid}.  After RTL, each
intermediate language is used to get closer to the assembly language of the
architecture, performing operations such as register allocation and making sure
that the stack variables are correctly aligned.

\subsection{CompCertSSA}%
\label{sec:bg:compcertssa}

\index{CompCertSSA}CompCertSSA is an extension of CompCert with an additional
\gls{SSA} intermediate language.  This language enforces \gls{SSA} properties
and therefore allows for many convenient proofs about optimisations performed on
this intermediate language, as many assumptions about variables can be made when
these are encountered.  The main interesting porperty that arises from using
\gls{SSA} is the \emph{equational lemma}, stating that given a variable, if it
was assigned by an operation that does not depend on memory, then loading the
destination value of that variable is the same as recomputing the value of that
variable with the current context.

Given a well formed SSA program $p$, a reachable state
$\Sigma\ s\ f\ \sigma\ R\ M$, a memory independent operation which was defined
at a node $d$ as $\mono{Iop}\ \mathit{op}\ \vec{a}\ x\ n$ assuming that $\sigma$
is dominated by $d$ ($p \le_{d} d$), then the following equation holds:

\begin{equation}\label{eq:equational-lemma}
  \left(\mathit{op}, \vec{a}\right) \Downarrow (R, M) = \left\lfloor R[x] \right\rfloor
\end{equation}

This is an important lemma as it essentially allows one to know the value of a
register as long as one knows that its assignment dominates the current node and
one knows what expressions it was assigned.

\section{Summary}%
\label{sec:bg:summary}

%%% Local Variables:
%%% mode: latex
%%% TeX-master: "../thesis"
%%% TeX-engine: luatex
%%% End:
