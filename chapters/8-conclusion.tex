\chapter{Conclusion}%
\label{sec:conclusion}

\section{Coq mechanisation}

\begin{table}
  \centering
  \captionabove{Statistics about the numbers of lines of code in the proof and
    implementation of \vericert{}, counted using \texttt{coqwc}.}%
  \label{tab:proof_statistics}
  \begin{tabular}{lrrrrr}
    \toprule
    & \textbf{OCaml} & \textbf{Spec} & \textbf{Proofs} & \textbf{Total}\\
    \midrule
    % ../common/DecEq.v ../common/Errormonad.v ../common/Optionmonad.v
    % ../common/Statemonad.v ../common/Monad.v ../common/NonEmpty.v
    % ../common/Vericertlib.v Array.v HashTree.v AssocMap.v
    {Data structures and libraries}       & --- & 1099 &  771 &  1870 \\
    % ../common/IntegerExtra.v ../common/ZExtra.v ValueInt.v
    {Integers and values}                 & --- &  393 &  520 &   913 \\
    % Gible.v GiblePar.v GibleSeq.v
    {\rtlblock{} and \rtlpar{} semantics} &  --- & 748 &  286 &  1034 \\
    % DeadBlocks.v DeadBlocksproof.v CondElim.v CondElimproof.v IfConversion.v
    % IfConversionproof.v GibleSeqgen.v GibleSeqgenproof.v
    {Hyperblock generation}               &  --- & 1716 & 1820 &  3536 \\
    % Schedule.ml Abstr.v AbstrSemIdent.v GiblePargen.v GiblePargenproof.v
    % GiblePargenproofBackward.v GiblePargenproofCommon.v
    % GiblePargenproofEquiv.v GiblePargenproofEvaluable.v
    % GiblePargenproofForward.v Sat.v
    {Hyperblock scheduling}               & 1083 & 3939 & 5597 & 10619 \\
    % Common.v Coqlib.v Errors.v HashTree.v Maps.v Predicate.v Smtpredicate.v
    {three-valued logic validator}            & 1276 & 4054 & 5040 & 10370 \\
    % DHTL.v
    {\htl{} semantics}                    &  --- &  249 &   31 &   280 \\
    % DHTLgen.v DHTLgenproof.v DHTLgenproof0.v GibleSubPargen.v GibleSubPar.v
    % GibleSubPargenproof.v
    {\htl{} generation}                   &  --- & 2248 & 2996 &  5244 \\
    % DMemorygen.v ClockMemory.v ClockMemoryproof.v
    {BRAM generation}                     &  --- & 1867 & 2890 &  4757 \\
    % ClockRegisters.v ClockRegistersproof.v
    {forward substitution}                &  --- &  481 &  425 &  906 \\
    % Verilog.v
    {Verilog semantics}                   &  --- &  628 &  124 &  752 \\
    % Veriloggen.v Veriloggenproof.v
    {Verilog generation}                  &  --- &  261 &  283 &  544 \\
    % Compiler.v Print*.ml
    {Top-level driver, pretty printers}   & 1404 &  273 &  223 & 1900 \\
    \midrule
    \textbf{Total}                        & 3763 & 17956 & 21006 & 42725 \\
    \bottomrule
  \end{tabular}
\end{table}

The lines of code for the implementation and proof of \vericert{} can be found
in \cref{tab:proof_statistics}.  Overall, it took about 3 person-years to build
\vericert{} -- about 6 person-months on implementation and 30 person-months on
proofs.  The largest proofs were by far the scheduling proof and the
three-valued logic validator.  The scheduling proof was difficult, and different
attempts with different proof styles were needed to complete it.  The main
difficulty was minimising the need for the three-valued validator, as initially
it appeared to be unnecessary.  In the end, it was needed to prove the
equivalence of predicates in the absence of structural equality.  This led to
the proof of the three-valued logic validator.  Even though it looks like a very
large proof, it was mainly quite straight-forward and purely technical, taking
roughly 1.5 months to complete.  The second hardest proof was the correctness
proof for the \htl{} generation, which required equivalence proofs between all
integer operations supported by \compcert{} and those supported in hardware.  A
large percentage of the proof is dedicated to the load and store instruction
translation.  These were tedious to prove correct because of the substantial
difference between the memory models used, and the need to prove properties such
as stores outside of the allocated memory being undefined, so that a finite
array could be used. In addition to that, since pointers in \htl{} and Verilog
are represented as integers, instead of as a separate \enquote{pointer} type
like in the \compcert{} semantics, one had to show that the integer arithmetic
was correct with respect to the pointer arithmetic in CompCert.  Many new
theorems had to be proven about them in \vericert{} to prove the conversion from
pointer to integer.  Moreover, another large proof in the back end describing
the correct \gls{BRAM} generation includes many proofs about the extensional
equality of array operations, such as merging arrays with different assignments.
Due to the negative edge implying that two merges take place every clock cycle,
it can become tedious to handle merges of changes performed during each of the
clock edges, as the extensional equality over merges needs to be specified for
each possible place the register assignment could have taken place.

\section{Limitations and Future Work}

There are various limitations in \vericert{} compared to other HLS tools due to
the fact that our main focus was on formally verifying the translation from
\rtl{} to Verilog.  In this section, we outline the current limitations of our
tool and suggest how they can be overcome in future work, first describing
limitations to the generated hardware, and then describing the limitations on
the software input that \vericert{} accepts.

\subsection{Limitations to the generated hardware}

\subsubsection{Lack of support for external \glsfmtshort{IPcore}}

\Glspl{IPcore} and other external hardware cores are often used by \gls{HLS}
tools to implement specific operations efficiently.  One example would be the
implementation of a division core, which can be designed as a fully pipelined
hardware design, executing different stages of an operation in parallel.  This
is in contrast to using the default division operation that generates the
division circuit in a single clock cycle.  This means that the otherwise
long-running division operation can be performed over multiple, shorter cycles,
leaving the overall maximum frequency of the design unchanged.  In \htl{},
\glspl{IPcore} could be represented in a similar fashion to load and store
instructions, by using wires to communicate with an abstract computation block
modelled in \htl{} and could later be replaced by a hardware implementation.

Furthermore, support for \glspl{IPcore} with a specific interface, such as a
simple ready/finished interface, could be implemented in general, so that
hardware that follows this pattern could be directly integrated into Vericert.
Integrating a new core would only require a proof of correctness of the
specification and the Verilog implementation, as well as an equivalence with the
operation that it should be replacing.  This would allow for a more general
implementation of the memory interface, for example, making it possible to
implement both loads and stores in the usual positive edge of the clock, as well
as making it possible to pipeline these loads and stores.

\subsubsection{Lack of Loop Pipelining}

Another critical \gls{HLS} optimisation that is often integrated with the
instruction scheduling optimisation is loop pipelining, also known as modulo
scheduling.  This is an important optimisation for \gls{HLS}, making it possible
to execute different parts of different loop iterations in parallel.  The ideal
scenario is that the whole function can eventually be pipelined so that it can
accept a stream of inputs every clock cycle.  This type of hardware design is
currently not possible with Vericert.  However, there is work on proving
software loop pipelining correct in
CompCert~\cite{tristan10_simpl_verif_valid_softw_pipel}, which could be adapted
and extended to support generating hardware pipelines, by using predicated
execution~\cite{rau92_code_gener_schem_sched_loops} and a rotating register
file~\cite{rau92_regis_alloc_softw_pipel_loops} to remove the need for a
prologue and epilogue in the pipelined loop.

\subsubsection{Limitations with I/O}

\vericert{} is currently limited in terms of I/O because the main function
cannot accept any arguments if the Clight program is to be
well-formed.\footnote{Technically, \vericert{} (and indeed, \compcert{}) can
  compile main functions that have arbitrary arguments and will handle those
  inputs appropriately, but the correctness theorem offers no guarantees about
  such programs.} Moreover, external function calls that produce traces have not
been implemented yet, but these could enable the C program to read and write
values on a bus that is shared with various other components in the hardware
design.

\subsection{Limitations on the software input}

\subsubsection{Lack of support for global variables}

In \compcert{}, each global variable is stored in its own memory.  A
generalisation of our translation of the stack into a \gls{BRAM} block could
therefore translate global variables in the same manner.  This would require a
slight generalisation of pointers so that they store provenance information to
ensure that each pointer accesses the right \gls{BRAM}. It would also be
necessary to generalise the \gls{BRAM} interface so that it decodes the
provenance information and indexes the correct array.

\subsubsection{Other language restrictions}

C and Verilog handle signedness quite differently. By default, all operators and
registers in Verilog (and \htl{}) are unsigned, so to force an operation to
handle the bits as signed, both operators have to be forced to be
signed. Moreover, Verilog implicitly resizes expressions to the largest needed
size by default, which can affect the result of the computation.  This feature
is not supported by the Verilog semantics we adopted, so to match the semantics
to the behaviour of the simulator and synthesis tool, braces are placed around
all expressions to inhibit implicit resizing.  Instead, explicit resizing is
used in the semantics, and operations can only be performed on two registers
that have the same size.

Furthermore, equality checks between \emph{unsigned} variables are actually not
supported, because this requires supporting the comparison of pointers, which
should only be performed between pointers with the same provenance.  In
\vericert{} there is currently no way to determine the provenance of a pointer,
and it therefore cannot model the semantics of unsigned comparison in
\compcert{}. This is not a severe restriction in practice however, because in
the absence of dynamic allocation, equality comparison of pointers is rarely
needed, and equality comparison of integers can still be performed by casting
them both to signed integers.

Finally, the \texttt{mulhs} and \texttt{mulhu} instructions, which fetch the
upper bits of a 32-bit multiplication, are not translated by \vericert{},
because 64-bit numbers are not supported. These instructions are only generated
to optimise divisions by a constant that is not a power of two, so turning off
constant propagation will allow these programs to pass without error.

\subsection{The Future of Vericert}

\subsubsection{Designing Correct Hardware}

It would be interesting to use the Vericert correctness theorem with the
Verified Software Toolchain~\cite{appel11_verif_softw_toolc} to prove that the
hardware implements the same specification as the software.  Specifications
would become more interesting with more support for I/O, however, it could
already be used to prove the correctness of the output of a Vericert design.

In addition to that, it would be interesting to use a verified synthesis tool to
build a complete workflow from software to hardware.  \textcite{lööw21_lutsig}
already showed that a Vericert design with registers with a bit-width could be
synthesised into \glspl{LUT} using Lutsig.  One could then imagine trying to
prove that a specification that is proven to hold about the C code would also
hold about the netlist.  The main issue is that this proof can not be performed
in one theorem prover, as Lutsig is proven correct in HOL4.

\subsubsection{Interaction With Software and Hardware}

Another direction that could be explored is hardware/software co-design or the
interaction of \gls{HLS} designs with externally defined hardware in a more
modular fashion.  CompCertO~\cite{koenig21_compc} makes it possible to reason
about the interaction of separate programs with different semantics through the
C calling convention of CompCert.  It would be interesting to extend the
Vericert Verilog semantics with a calling convention that could be used to
interact and reason about the hardware at this level.  In this way, one could
either verify heterogeneous systems, or verify and reason about the interaction
of CompCert C designs destined to be synthesised into hardware with externally
verified hardware designs.

\section{Summary}

In conclusion, the need for a correct high-level synthesis led to the design of
Vericert, a verified \gls{HLS} tool based on CompCert.  We showed that with the
implementation of hyperblock scheduling, the performance of Vericert was around
$1.6\times$ performance of the unoptimised version of Bambu HLS, which is
promising, while being around $3.6\times$ slower than optimised Bambu HLS.  We
hope that Vericert will enable \gls{HLS} to be used in spaces where the
correctness of the hardware is paramount, for example for compiling
cryptographic algorithms from C, where performance might only be a secondary
concern.  This should still allow for efficient application-specific
accelerators, and together with a proof of correctness between the netlist and
the hardware design, a near end-to-end correctness theorem for the accelerator.

%%% Local Variables:
%%% mode: latex
%%% TeX-master: "../thesis"
%%% TeX-engine: luatex
%%% End:
