\chapter{Evaluation}%
\label{sec:evaluation}

\section{Performance Evaluation}
\label{sec:performance-comparison}

Our evaluation aims to answer the following research questions:

\begin{itemize}

\item {\textbf{RQ1:}} Does adding scheduling to Vericert lead to a significant improvement in the quality of the generated hardware (in terms of area and delay)?

\item {\textbf{RQ2:}} Is hyperblock scheduling better than na\"ive list scheduling?

\item {\textbf{RQ3:}} Does adding scheduling make Vericert competitive with unverified HLS tools?

\item {\textbf{RQ4:}} Did our design decisions (e.g. \cref{sec:thirdattempt}) lead to an acceptable compilation time?

\end{itemize}

\definecolor{colorVericertBase}{HTML}{66c2a5}
\definecolor{colorVericertList}{HTML}{fc8d62}
\definecolor{colorVericertHyper}{HTML}{8da0cb}
\definecolor{colorBambuNoOpt}{HTML}{e78ac3}
\definecolor{colorBambuDefault}{HTML}{bbbbbb}

\colorlet{colorVericertBaseLIGHT}{colorVericertBase!50!white}
\colorlet{colorVericertListLIGHT}{colorVericertList!50!white}
\colorlet{colorVericertHyperLIGHT}{colorVericertHyper!50!white}
\colorlet{colorBambuNoOptLIGHT}{colorBambuNoOpt!50!white}
\colorlet{colorBambuDefaultLIGHT}{colorBambuDefault!50!white}

\newcommand\BambuDefault{%
\setul{-1pt}{3pt}\setulcolor{colorBambuDefaultLIGHT}%
{\ul{\textsf{Bambu-default}}}}

\newcommand\BambuNoOpt{%
\setul{-1pt}{3pt}\setulcolor{colorBambuNoOptLIGHT}%
{\ul{\textsf{Bambu-no-opt}}}}

\newcommand\VericertBase{%
\setul{-1pt}{3pt}\setulcolor{colorVericertBaseLIGHT}%
{\ul{\textsf{Vericert-original}}}}

\newcommand\VericertList{%
\setul{-1pt}{3pt}\setulcolor{colorVericertListLIGHT}%
{\ul{\textsf{Vericert-list-scheduling}}}}

\newcommand\VericertHyper{%
\setul{-1pt}{3pt}\setulcolor{colorVericertHyperLIGHT}%
{\ul{\textsf{Vericert-hyperblock-scheduling}}}}

\begin{figure*}
  \centering
  \resizebox{\linewidth}{!}{% GNUPLOT: LaTeX picture with Postscript
\begingroup
  \makeatletter
  \providecommand\color[2][]{%
    \GenericError{(gnuplot) \space\space\space\@spaces}{%
      Package color not loaded in conjunction with
      terminal option `colourtext'%
    }{See the gnuplot documentation for explanation.%
    }{Either use 'blacktext' in gnuplot or load the package
      color.sty in LaTeX.}%
    \renewcommand\color[2][]{}%
  }%
  \providecommand\includegraphics[2][]{%
    \GenericError{(gnuplot) \space\space\space\@spaces}{%
      Package graphicx or graphics not loaded%
    }{See the gnuplot documentation for explanation.%
    }{The gnuplot epslatex terminal needs graphicx.sty or graphics.sty.}%
    \renewcommand\includegraphics[2][]{}%
  }%
  \providecommand\rotatebox[2]{#2}%
  \@ifundefined{ifGPcolor}{%
    \newif\ifGPcolor
    \GPcolortrue
  }{}%
  \@ifundefined{ifGPblacktext}{%
    \newif\ifGPblacktext
    \GPblacktexttrue
  }{}%
  % define a \g@addto@macro without @ in the name:
  \let\gplgaddtomacro\g@addto@macro
  % define empty templates for all commands taking text:
  \gdef\gplbacktext{}%
  \gdef\gplfronttext{}%
  \makeatother
  \ifGPblacktext
    % no textcolor at all
    \def\colorrgb#1{}%
    \def\colorgray#1{}%
  \else
    % gray or color?
    \ifGPcolor
      \def\colorrgb#1{\color[rgb]{#1}}%
      \def\colorgray#1{\color[gray]{#1}}%
      \expandafter\def\csname LTw\endcsname{\color{white}}%
      \expandafter\def\csname LTb\endcsname{\color{black}}%
      \expandafter\def\csname LTa\endcsname{\color{black}}%
      \expandafter\def\csname LT0\endcsname{\color[rgb]{1,0,0}}%
      \expandafter\def\csname LT1\endcsname{\color[rgb]{0,1,0}}%
      \expandafter\def\csname LT2\endcsname{\color[rgb]{0,0,1}}%
      \expandafter\def\csname LT3\endcsname{\color[rgb]{1,0,1}}%
      \expandafter\def\csname LT4\endcsname{\color[rgb]{0,1,1}}%
      \expandafter\def\csname LT5\endcsname{\color[rgb]{1,1,0}}%
      \expandafter\def\csname LT6\endcsname{\color[rgb]{0,0,0}}%
      \expandafter\def\csname LT7\endcsname{\color[rgb]{1,0.3,0}}%
      \expandafter\def\csname LT8\endcsname{\color[rgb]{0.5,0.5,0.5}}%
    \else
      % gray
      \def\colorrgb#1{\color{black}}%
      \def\colorgray#1{\color[gray]{#1}}%
      \expandafter\def\csname LTw\endcsname{\color{white}}%
      \expandafter\def\csname LTb\endcsname{\color{black}}%
      \expandafter\def\csname LTa\endcsname{\color{black}}%
      \expandafter\def\csname LT0\endcsname{\color{black}}%
      \expandafter\def\csname LT1\endcsname{\color{black}}%
      \expandafter\def\csname LT2\endcsname{\color{black}}%
      \expandafter\def\csname LT3\endcsname{\color{black}}%
      \expandafter\def\csname LT4\endcsname{\color{black}}%
      \expandafter\def\csname LT5\endcsname{\color{black}}%
      \expandafter\def\csname LT6\endcsname{\color{black}}%
      \expandafter\def\csname LT7\endcsname{\color{black}}%
      \expandafter\def\csname LT8\endcsname{\color{black}}%
    \fi
  \fi
    \setlength{\unitlength}{0.0500bp}%
    \ifx\gptboxheight\undefined%
      \newlength{\gptboxheight}%
      \newlength{\gptboxwidth}%
      \newsavebox{\gptboxtext}%
    \fi%
    \setlength{\fboxrule}{0.5pt}%
    \setlength{\fboxsep}{1pt}%
    \definecolor{tbcol}{rgb}{1,1,1}%
\begin{picture}(10800.00,7200.00)%
    \gplgaddtomacro\gplbacktext{%
      \csname LTb\endcsname%%
      \put(518,5449){\makebox(0,0)[r]{\strut{}1}}%
      \csname LTb\endcsname%%
      \put(518,5766){\makebox(0,0)[r]{\strut{}2}}%
      \csname LTb\endcsname%%
      \put(518,5951){\makebox(0,0)[r]{\strut{}3}}%
      \csname LTb\endcsname%%
      \put(518,6185){\makebox(0,0)[r]{\strut{}5}}%
      \csname LTb\endcsname%%
      \put(518,6501){\makebox(0,0)[r]{\strut{}10}}%
      \csname LTb\endcsname%%
      \put(518,6920){\makebox(0,0)[r]{\strut{}25}}%
    }%
    \gplgaddtomacro\gplfronttext{%
      \csname LTb\endcsname%%
      \put(161,6202){\rotatebox{-270}{\makebox(0,0){\strut{}Relative execution time}}}%
    }%
    \gplgaddtomacro\gplbacktext{%
      \csname LTb\endcsname%%
      \put(518,3477){\makebox(0,0)[r]{\strut{}1}}%
      \csname LTb\endcsname%%
      \put(518,3811){\makebox(0,0)[r]{\strut{}2}}%
      \csname LTb\endcsname%%
      \put(518,4007){\makebox(0,0)[r]{\strut{}3}}%
      \csname LTb\endcsname%%
      \put(518,4253){\makebox(0,0)[r]{\strut{}5}}%
      \csname LTb\endcsname%%
      \put(518,4587){\makebox(0,0)[r]{\strut{}10}}%
      \csname LTb\endcsname%%
      \put(518,5029){\makebox(0,0)[r]{\strut{}25}}%
    }%
    \gplgaddtomacro\gplfronttext{%
      \csname LTb\endcsname%%
      \put(161,4228){\rotatebox{-270}{\makebox(0,0){\strut{}Relative cycle count}}}%
    }%
    \gplgaddtomacro\gplbacktext{%
      \csname LTb\endcsname%%
      \put(616,1546){\makebox(0,0)[r]{\strut{}0.7}}%
      \csname LTb\endcsname%%
      \put(616,1855){\makebox(0,0)[r]{\strut{}1.0}}%
      \csname LTb\endcsname%%
      \put(616,2455){\makebox(0,0)[r]{\strut{}2.0}}%
      \csname LTb\endcsname%%
      \put(616,2806){\makebox(0,0)[r]{\strut{}3.0}}%
      \csname LTb\endcsname%%
      \put(616,3055){\makebox(0,0)[r]{\strut{}4.0}}%
      \csname LTb\endcsname%%
      \put(888,1448){\rotatebox{-90}{\makebox(0,0)[l]{\strut{}2mm}}}%
      \csname LTb\endcsname%%
      \put(1237,1448){\rotatebox{-90}{\makebox(0,0)[l]{\strut{}3mm}}}%
      \csname LTb\endcsname%%
      \put(1586,1448){\rotatebox{-90}{\makebox(0,0)[l]{\strut{}adi}}}%
      \csname LTb\endcsname%%
      \put(1935,1448){\rotatebox{-90}{\makebox(0,0)[l]{\strut{}atas}}}%
      \csname LTb\endcsname%%
      \put(2284,1448){\rotatebox{-90}{\makebox(0,0)[l]{\strut{}bicg}}}%
      \csname LTb\endcsname%%
      \put(2633,1448){\rotatebox{-90}{\makebox(0,0)[l]{\strut{}cholesky}}}%
      \csname LTb\endcsname%%
      \put(2982,1448){\rotatebox{-90}{\makebox(0,0)[l]{\strut{}covariance}}}%
      \csname LTb\endcsname%%
      \put(3331,1448){\rotatebox{-90}{\makebox(0,0)[l]{\strut{}doitgen}}}%
      \csname LTb\endcsname%%
      \put(3680,1448){\rotatebox{-90}{\makebox(0,0)[l]{\strut{}durbin}}}%
      \csname LTb\endcsname%%
      \put(4029,1448){\rotatebox{-90}{\makebox(0,0)[l]{\strut{}fdtd-2d}}}%
      \csname LTb\endcsname%%
      \put(4378,1448){\rotatebox{-90}{\makebox(0,0)[l]{\strut{}floyd-warshall}}}%
      \csname LTb\endcsname%%
      \put(4727,1448){\rotatebox{-90}{\makebox(0,0)[l]{\strut{}gemm}}}%
      \csname LTb\endcsname%%
      \put(5076,1448){\rotatebox{-90}{\makebox(0,0)[l]{\strut{}gemver}}}%
      \csname LTb\endcsname%%
      \put(5425,1448){\rotatebox{-90}{\makebox(0,0)[l]{\strut{}gesummv}}}%
      \csname LTb\endcsname%%
      \put(5774,1448){\rotatebox{-90}{\makebox(0,0)[l]{\strut{}heat-3d}}}%
      \csname LTb\endcsname%%
      \put(6123,1448){\rotatebox{-90}{\makebox(0,0)[l]{\strut{}jacobi-1d}}}%
      \csname LTb\endcsname%%
      \put(6472,1448){\rotatebox{-90}{\makebox(0,0)[l]{\strut{}jacobi-2d}}}%
      \csname LTb\endcsname%%
      \put(6821,1448){\rotatebox{-90}{\makebox(0,0)[l]{\strut{}lu}}}%
      \csname LTb\endcsname%%
      \put(7170,1448){\rotatebox{-90}{\makebox(0,0)[l]{\strut{}ludcmp}}}%
      \csname LTb\endcsname%%
      \put(7519,1448){\rotatebox{-90}{\makebox(0,0)[l]{\strut{}mvt}}}%
      \csname LTb\endcsname%%
      \put(7868,1448){\rotatebox{-90}{\makebox(0,0)[l]{\strut{}nussinov}}}%
      \csname LTb\endcsname%%
      \put(8217,1448){\rotatebox{-90}{\makebox(0,0)[l]{\strut{}seidel-2d}}}%
      \csname LTb\endcsname%%
      \put(8566,1448){\rotatebox{-90}{\makebox(0,0)[l]{\strut{}symm}}}%
      \csname LTb\endcsname%%
      \put(8915,1448){\rotatebox{-90}{\makebox(0,0)[l]{\strut{}syr2k}}}%
      \csname LTb\endcsname%%
      \put(9264,1448){\rotatebox{-90}{\makebox(0,0)[l]{\strut{}syrk}}}%
      \csname LTb\endcsname%%
      \put(9613,1448){\rotatebox{-90}{\makebox(0,0)[l]{\strut{}trisolv}}}%
      \csname LTb\endcsname%%
      \put(9962,1448){\rotatebox{-90}{\makebox(0,0)[l]{\strut{}trmm}}}%
      \csname LTb\endcsname%%
      \put(10311,1448){\rotatebox{-90}{\makebox(0,0)[l]{\strut{}median}}}%
    }%
    \gplgaddtomacro\gplfronttext{%
      \csname LTb\endcsname%%
      \put(161,2300){\rotatebox{-270}{\makebox(0,0){\strut{}Relative area}}}%
      \csname LTb\endcsname%%
      \put(9730,2896){\makebox(0,0)[r]{\strut{}\footnotesize Vericert-original}}%
      \csname LTb\endcsname%%
      \put(9730,2720){\makebox(0,0)[r]{\strut{}\footnotesize Vericert-list-scheduling}}%
      \csname LTb\endcsname%%
      \put(9730,2545){\makebox(0,0)[r]{\strut{}\footnotesize Vericert-hyperblock-scheduling}}%
      \csname LTb\endcsname%%
      \put(9730,2369){\makebox(0,0)[r]{\strut{}\footnotesize Bambu-no-opt}}%
    }%
    \gplbacktext
    \put(0,0){\includegraphics[width={540.00bp},height={360.00bp}]{bar-plot}}%
    \gplfronttext
  \end{picture}%
\endgroup
}
  \caption[Results of simulating and synthesising the PolyBench/C benchmark suite using a range of HLS tools. All figures are relative to Bambu.]{Results of simulating and synthesising the PolyBench/C benchmark suite using a range of HLS tools. All figures are relative to \BambuDefault{}.}%
  \label{fig:list-against-hyper-scheduling}
\end{figure*}

\paragraph{Experimental setup}
Following \textcite{herklotz21_fvhls} and
\textcite{six22_formal_verif_super_sched}, we evaluate our work using
PolyBench/C~\cite{pouchet20_polyb_c}. For each benchmark, the resulting Verilog
hardware design was simulated using Verilator to get the total cycle count. Each
design was synthesised, placed, and routed onto a Xilinx series 7 FPGA (part
number: \mono{xc7z020clg484-1}) using Vivado to get its total area and its
maximum frequency.  We then calculated
$\text{total execution time} = \frac{\text{total clock cycles}}{\text{maximum
    frequency}}$.  We ensured that every design met the timing constraints of a
100MHz clock.

\paragraph{Answering RQ1}
To assess whether adding scheduling to Vericert leads to better hardware designs, \cref{fig:list-against-hyper-scheduling} compares the hardware produced by original Vericert (\VericertBase{}) with that produced when hyperblock scheduling is enabled (\VericertHyper{}). We see that, on average, hyperblock scheduling leads to hardware that requires only 0.46$\times$ the cycle count (middle plot). This is unsurprising given that original Vericert only executed a single instruction per clock cycle. In terms of area (bottom plot), hyperblock scheduling has, on average, a slight increase in area.

\paragraph{Answering RQ2}
Hyperblock scheduling is considerably more complicated to implement and verify than list scheduling, as it requires if-conversion to combine basic blocks into hyperblocks, as well as predicate-aware scheduling. If we omit if-conversion entirely (hence avoiding predication too), we obtain list scheduling as a special case. Does hyperblock scheduling yield enough of a performance improvement over list scheduling to justify its additional complexity?

To answer this, \cref{fig:list-against-hyper-scheduling} measures the hardware produced by Vericert with list scheduling (\VericertList{}). On average, list scheduling leads to hardware that requires 0.51$\times$ the cycle count compared to \VericertBase{}, which is 1.1$\times$ the cycle count compared to \VericertHyper{}. We expect hyperblock scheduling to extend its small lead over list scheduling once the heuristics that guide if-conversion are improved.  In particular, our predictions of the latency of predicated instructions are currently quite conservative to ensure that timing constraints are met; improving these estimates is an active research area~\cite{tan15_mappin_lut_fpgas,rizzi23_iterat_method_mappin_aware_frequen,wang23_mapbuf,ustun20_accur_fpga_hls,zheng14_fast_effec_placem_routin_direc}.

In terms of area, we see that \VericertList{} leads to the smallest hardware designs. This can be attributed to the downstream logic synthesis tool being able to save area by optimising chained operations, such as multiply--accumulate, while not having to handle the predicates that are introduced with \VericertHyper{}.

\paragraph{Answering RQ3}
To assess how \VericertHyper{} fares against unverified HLS tools, we compare it
against the state-of-the-art open-source HLS tool
Bambu~\cite[]{ferrandi21_bambu}. We use Bambu in two modes: one where all
default optimisations are enabled (\BambuDefault{}), and one where as many
optimisations as possible are disabled (\BambuNoOpt{}). Note that several
\enquote{optimisations} are built into Bambu and cannot be disabled, such as
list scheduling and loop flattening.

All the bars in \cref{fig:list-against-hyper-scheduling} are relative to \BambuDefault. The pink bars show \BambuNoOpt. We see that although \VericertHyper{} is well behind \BambuDefault{} (its designs require 3$\times$ the cycle count), it performs comparably to \BambuNoOpt{} (1.04$\times$ the cycle count), which is encouraging because \VericertHyper{} and \BambuNoOpt{} have similar feature sets.

\paragraph{Answering RQ4}
To assess whether \VericertHyper{} has acceptable compilation times, we also
compare it against Bambu.  Compilation times did not deviate for Bambu, all of
them being around 3s mainly due to long startup costs. \VericertHyper{} compiled
each benchmark in 0.9s, also without much variation, showing that verification
was not overly costly.  As for whether our design decisions led to these
compilation times: we remark that if we disable the \enquote{final-state
  predicates} innovation that we introduced in \cref{sec:thirdattempt}, none of
the benchmarks compile within a few minutes and eventually the machine runs out
of memory.

% \Cref{fig:list-against-hyper-scheduling} shows the final results relative to the base version of Vericert.  First, the relative cycle counts between each tool shows that list scheduling has 0.59$\times$ the number of cycles compared to base Vericert and hyperblock scheduling has 0.56$\times$ the number of cycles compared to base Vericert, showing that scheduling instructions provides a large improvement compared to the total number of cycles of base Vericert.  Bambu with optimisations turned-off has around 0.42$\times$ the number of cycles, and with optimisations has 0.14$\times$ the number of cycles, taking drastically fewer cycles.  One outlier here is the jacobi-1d benchmark, where only optimised Bambu finds a way to reduce the number of cycles.  This is because it is a very small benchmark with a single loop, which cannot be optimised by the scheduling algorithms and requires more advanced loop optimisations such as loop pipelining.

% However, looking at the relative execution time is a bit surprising, because on average, the hyperblock scheduling algorithm only performs as well as base Vericert, whereas the list scheduling algorithm performs much better.  This is because the operation chaining heuristics used did not work consistently for the hyperblock scheduling pass, therefore reducing the maximum operating frequency dramatically in some cases.  This is something that needs to be addressed in the heuristics used to perform the if-conversion, but also in the latency constraints in the scheduler.  Interestingly, however, the output of the list scheduling algorithm is around 14\% faster than unoptimised Bambu.  Again, optimised Bambu optimises the benchmarks much further, but also has a slightly higher maximum frequency, bringing the gap down a bit compared to the total cycle counts.

% Finally, looking at area, list scheduling actually also reduces the area
% compared to base Vericert, which is mainly due to the synthesis tool being
% able to optimise chained operations, such as multiply-accumulate operations,
% further.  However, because of the addition of predicates in hyperblock
% scheduling, the area is similar to base Vericert.  This area is similar to
% unoptimised Bambu, however, optimised Bambu achieves 0.6$\times$ the area.

%%% Local Variables:
%%% mode: latex
%%% TeX-master: "../thesis"
%%% TeX-engine: luatex
%%% End:
