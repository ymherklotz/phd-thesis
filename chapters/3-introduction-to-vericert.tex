\chapter{Introduction to Vericert}

This section describes the main architecture of the HLS tool, and the way in which the Verilog back end was added to \compcert{}.  This section also covers an example of converting a simple C program into hardware, expressed in the Verilog language.

\subsection{Main Design Decisions}

\paragraph{Choice of source language}
C was chosen as the source language as it remains the most common source language amongst production-quality HLS tools~\cite{canis11_legup, xilinx20_vivad_high_synth, intel_hls, bambu_hls}. This, in turn, may be because it is ``[t]he starting point for the vast majority of algorithms to be implemented in hardware''~\cite{5522874}, lending a degree of practicality.
The availability of \compcert{}~\cite{leroy09_formal_verif_realis_compil} also provides a solid basis for formally verified C compilation.
%Since a lot of existing code for HLS is written in C, supporting C as an input language, rather than a custom domain-specific language, means that \vericert{} is more practical.
%An alternative was to support LLVM IR as an input language, however, to get a full work flow from a higher level language to hardware, a front end for that language to LLVM IR would also have to be verified. \JW{Maybe save LLVM for the `Choice of implementation language'?}
We considered Bluespec~\cite{nikhil04_blues_system_veril}, but decided that although it ``can be classed as a high-level language''~\cite{greaves_note}, it is too hardware-oriented to be suitable for traditional HLS.
We also considered using a language with built-in parallel constructs that map well to parallel hardware, such as occam~\cite{page91_compil_occam}, Spatial~\cite{spatial} or Scala~\cite{chisel}.
% However, this would not qualify as being HLS due to the manual parallelism that would have to be performed. \JW{I don't think the presence of parallelism stops it being proper HLS.}
%\JP{I think I agree with Yann here, but it could be worded better. At any rate not many people have experience writing what is essentially syntactic sugar over a process calculus.}
%\JW{I mean: there are plenty of software languages that involve parallel constructs. Anyway, perhaps we can just dismiss occam for being too obscure.}


\paragraph{Choice of target language}
Verilog~\cite{06_ieee_stand_veril_hardw_descr_languag} is an HDL that can be synthesised into logic cells which can either be placed onto a field-programmable gate array (FPGA) or turned into an application-specific integrated circuit (ASIC).  Verilog was chosen as the output language for \vericert{} because it is one of the most popular HDLs and there already exist a few formal semantics for it that could be used as a target~\cite{loow19_verif_compil_verif_proces, meredith10_veril}.  Bluespec, previously ruled out as a source language, is another possible target and there exists a formally verified translation to circuits using K\^{o}ika~\cite{bourgeat20_essen_blues}. %\JP{This needs an extra comment maybe?}\YH{Maybe about bluespec not being an ideal target language because it's quite high-level?} % but targeting this language would not be trivial as it is not meant to be targeted by an automatic tool, instead strives to a formally verified high-level hardware description language instead.

%\JW{Can we mention one or two alternatives that we considered? Bluespec or Chisel or one of Adam Chlipala's languages, perhaps?}

\paragraph{Choice of implementation language}
We chose Coq as the implementation language because of its mature support for code extraction; that is, its ability to generate OCaml programs directly from the definitions used in the theorems.
We note that other authors have had some success reasoning about the HLS process using other theorem provers such as Isabelle~\cite{ellis08}.
\compcert{}~\cite{leroy09_formal_verif_realis_compil} was chosen as the front end because it has a well established framework for simulation proofs about intermediate languages, and it already provides a validated C parser~\cite{jourdan12_valid_lr_parser}.
The Vellvm framework~\cite{zhao12_formal_llvm_inter_repres_verif_progr_trans} was also considered because several existing HLS tools are already LLVM-based, but additional work would be required to support a high-level language like C as input.
The .NET framework has been used as a basis for other HLS tools, such as Kiwi~\cite{kiwi}, and LLHD~\cite{schuiki20_llhd} has been recently proposed as an intermediate language for hardware design, but neither are suitable for us because they lack formal semantics.

\begin{figure}
  \centering
  % JW: resizebox is associated with puppies dying
  %\resizebox{0.85\textwidth}{!}{
  \begin{tikzpicture}
    [language/.style={fill=white,rounded corners=3pt,minimum height=7mm},
    continuation/.style={},
    linecount/.style={rounded corners=3pt,dashed}]
    \fill[compcert,rounded corners=3pt] (-1,-0.5) rectangle (9.9,2);
    \fill[formalhls,rounded corners=3pt] (-1,-1) rectangle (9.9,-2.4);
    %\draw[linecount] (-0.95,-0.45) rectangle (3.6,1);
    %\draw[linecount] (4,-0.45) rectangle (7.5,1);
    \node[language] at (-0.3,0) (clight) {Clight};
    \node[continuation] at (1,0) (conta) {$\cdots$};
    \node[language] at (2.7,0) (cminor) {CminorSel};
    \node[language] at (4.7,0) (rtl) {3AC};
    \node[language] at (6.2,0) (ltl) {LTL};
    \node[language,anchor=west] at (8.4,0) (aarch) {aarch64};
    \node[language,anchor=west] at (8.4,0.8) (x86) {x86};
    \node[continuation,anchor=west] at (8.4,1.4) (backs) {$\cdots$};
    \node[continuation] at (7.3,0) (contb) {$\cdots$};
    \node[language] at (4.7,-1.5) (htl) {HTL};
    \node[language] at (6.7,-1.5) (verilog) {Verilog};
    \node[anchor=west] at (-0.9,1.6) {\textbf{\compcert{}}};
    \node[anchor=west] at (-0.9,-1.4) {\textbf{\vericert{}}};
    %%\node[anchor=west] at (-0.9,0.7) {\small $\sim$ 27 kloc};
    %%\node[anchor=west] at (4.1,0.7) {\small $\sim$ 46 kloc};
    %%\node[anchor=west] at (2,-1.5) {\small $\sim$ 17 kloc};
    \node[align=center] at (3.2,-2) {\footnotesize RAM\\[-0.5em]\footnotesize insertion};
    \draw[->,thick] (clight) -- (conta);
    \draw[->,thick] (conta) -- (cminor);
    \draw[->,thick] (cminor) -- (rtl);
    \draw[->,thick] (rtl) -- (ltl);
    \draw[->,thick] (ltl) -- (contb);
    \draw[->,thick] (contb) -- (aarch);
    \draw[->,thick] (contb) to [out=0,in=200] (x86.west);
    \draw[->,thick] (contb) to [out=0,in=190] (backs.west);
    \draw[->,thick] (rtl) -- (htl);
    \draw[->,thick] (htl) -- (verilog);
    \draw[->,thick] (htl.west) to [out=180,in=150] (4,-2.2) to [out=330,in=270] (htl.south);
  \end{tikzpicture}%}
%  \alt{The compilation flow of \compcert{}, showing where \vericert{} branches off.}
  \caption{\vericert{} as a Verilog back end to \compcert{}.}%
  \label{fig:rtlbranch}
\end{figure}

\paragraph{Architecture of \vericert{}}
The main work flow of \vericert{} is given in Fig.~\ref{fig:rtlbranch}, which shows those parts of the translation that are performed in \compcert{}, and those that have been added.  This includes translations to two new intermediate languages added in \vericert{}, HTL and Verilog, as well as an additional optimisation pass labelled as ``RAM insertion''.

\def\numcompcertlanguages{ten}

\compcert{} translates Clight\footnote{A deterministic subset of C with pure expressions.} input into assembly output via a sequence of intermediate languages; we must decide which of these \numcompcertlanguages{} languages is the most suitable starting point for the HLS-specific translation stages.

We select CompCert's three-address code (3AC)\footnote{This is known as register transfer language (RTL) in the \compcert{} literature. `3AC' is used in this paper instead to avoid confusion with register-transfer level (RTL), which is another name for the final hardware target of the HLS tool.} as the starting point. Branching off \emph{before} this point (at CminorSel or earlier) denies \compcert{} the opportunity to perform optimisations such as constant propagation and dead-code elimination, which, despite being designed for software compilers, have been found useful in HLS tools as well~\cite{cong+11}. And if we branch off \emph{after} this point (at LTL or later) then \compcert{} has already performed register allocation to reduce the number of registers and spill some variables to the stack; this transformation is not required in HLS because there are many more registers available, and these should be used instead of RAM whenever possible. %\JP{``\compcert{} performs register allocation during the translation to LTL, with some registers spilled onto the stack: this is unnecessary in HLS since as many registers as are required may be described in the output RTL.''} \JP{Maybe something about FPGAs being register-dense (so rarely a need to worry about the number of flops)?}

3AC is also attractive because it is the closest intermediate language to LLVM IR, which is used by several existing HLS compilers. %\JP{We already ruled out LLVM as a starting point, so this seems like it needs further qualification.}\YH{Well not because it's not a good starting point, but the ecosystem in Coq isn't as good.  I think it's still OK here to say that being similar to LLVM IR is an advantage?}
It has an unlimited number of pseudo-registers, and is represented as a control flow graph (CFG) where each instruction is a node with links to the instructions that can follow it. One difference between LLVM IR and 3AC is that 3AC includes operations that are specific to the chosen target architecture; we chose to target the x86\_32 back end because it generally produces relatively dense 3AC thanks to the availability of complex addressing modes.% reducing cycle counts in the absence of an effective scheduling approach.

\subsection{An Introduction to Verilog}

This section will introduce Verilog for readers who may not be familiar with the language, concentrating on the features that are used in the output of \vericert{}.  Verilog is a hardware description language (HDL) and is used to design hardware ranging from complete CPUs that are eventually produced as integrated circuits, to small application-specific accelerators that are placed on FPGAs.  Verilog is a popular language because it allows for fine-grained control over the hardware, and also provides high-level constructs to simplify development.

Verilog behaves quite differently to standard software programming languages due to it having to express the parallel nature of hardware.  The basic construct to achieve this is the always-block, which is a collection of assignments that are executed every time some event occurs.  In the case of \vericert{}, this event is either a positive (rising) or a negative (falling) clock edge.  All always-blocks triggering on the same event are executed in parallel. Always-blocks can also express control-flow using if-statements and case-statements.
%\NR{Might be useful to talk about registers must be updated only within an always-block.} \JW{That's important for Verilog programming in general, but is it necessary for understanding this paper?}\YH{Yeah, I don't think it is too important for this section.}

\begin{figure}
  \centering
  \begin{subfigure}{0.55\linewidth}
\begin{minted}[linenos,xleftmargin=20pt,fontsize=\footnotesize]{verilog}
module main(input rst, input y, input clk,
            output reg z);
  reg tmp, state;
  always @(posedge clk)
    case (state)
      1'b0: tmp <= y;
      1'b1: begin tmp <= 1'b0; z <= tmp; end
    endcase
  always @(posedge clk)
    if (rst) state <= 1'b0;
    else case (state)
      1'b0: if (y) state <= 1'b1;
            else state <= 1'b0;
      1'b1: state <= 1'b0;
    endcase
endmodule
\end{minted}
  \end{subfigure}\hfill%
  \begin{subfigure}{0.45\linewidth}
    \centering
    \begin{tikzpicture}
      \node[draw,circle,inner sep=6pt] (s0) at (0,0) {$S_{\mathit{start}} / \texttt{x}$};
      \node[draw,circle,inner sep=8pt] (s1) at (1.5,-3) {$S_{1} / \texttt{1}$};
      \node[draw,circle,inner sep=8pt] (s2) at (3,0) {$S_{0} / \texttt{1}$};
      \node (s2s) at ($(s2.west) + (-0.3,1)$) {\texttt{00}};
      \node (s2ss) at ($(s2.east) + (0.3,1)$) {\texttt{1x}};
      \draw[-{Latex[length=2mm,width=1.4mm]}] ($(s0.west) + (-0.3,1)$) to [out=0,in=120] (s0);
      \draw[-{Latex[length=2mm,width=1.4mm]}] (s0)
           to [out=-90,in=150] node[midway,left] {\texttt{01}} (s1);
      \draw[-{Latex[length=2mm,width=1.4mm]}] (s1)
      to [out=80,in=220] node[midway,left] {\texttt{xx}} (s2);
      \draw[-{Latex[length=2mm,width=1.4mm]}] (s2)
      to [out=260,in=50] node[midway,right] {\texttt{01}} (s1);
      \draw[-{Latex[length=2mm,width=1.4mm]}] (s2)
      to [out=120,in=40] ($(s2.west) + (-0.3,0.7)$) to [out=220,in=170] (s2);
      \draw[-{Latex[length=2mm,width=1.4mm]}] (s2)
      to [out=60,in=130] ($(s2.east) + (0.3,0.7)$) to [out=310,in=10] (s2);
    \end{tikzpicture}
  \end{subfigure}
%  \alt{Verilog code of a state machine, and its equivalent state machine diagram.}
  \caption{A simple state machine implemented in Verilog, with its diagrammatic representation on the right. The \texttt{x} stands for ``don't care'' and each transition is labelled with the values of the inputs \texttt{rst} and \texttt{y} that trigger the transition.  The output that will be produced is shown in each state.}%
  \label{fig:tutorial:state_machine}
\end{figure}


A simple state machine can be implemented as shown in Fig.~\ref{fig:tutorial:state_machine}.
At every positive edge of the clock (\texttt{clk}), both of the always-blocks will trigger simultaneously.  The first always-block controls the values in the register \texttt{tmp} and the output \texttt{z}, while the second always-block controls the next state the state machine should go to.  When the \texttt{state} is 0, \texttt{tmp} will be assigned to the input \texttt{y} using nonblocking assignment, denoted by \texttt{<=}.  Nonblocking assignment assigns registers in parallel at the end of the clock cycle, rather than sequentially throughout the always-block. In the second always-block, the input \texttt{y} will be checked, and if it's high it will move on to the next state, otherwise it will stay in the current state.  When \texttt{state} is 1, the first always-block will reset the value of \texttt{tmp} and then set \texttt{z} to the original value of \texttt{tmp}, since nonblocking assignment does not change its value until the end of the clock cycle.  Finally, the last always-block will set the state to 0 again.

\begin{figure}
  \centering
  \begin{subfigure}[b]{0.3\linewidth}
    \begin{subfigure}[t]{1\linewidth}
\begin{minted}[fontsize=\footnotesize,linenos,xleftmargin=20pt]{c}
int main() {
    int x[2] = {3, 6};
    int i = 1;
    return x[i];
}
\end{minted}
      \caption{Example C code passed to \vericert{}.}\label{fig:accumulator_c}
    \end{subfigure}\\\vspace{3em}
    \begin{subfigure}[b]{1\linewidth}
\begin{minted}[fontsize=\footnotesize,linenos,xleftmargin=20pt]{c}
main() {
    x5 = 3
    int32[stack(0)] = x5
    x4 = 6
    int32[stack(4)] = x4
    x1 = 1
    x3 = stack(0) (int)
    x2 = int32[x3 + x1
               * 4 + 0]
    return x2
}
\end{minted}
      \caption{3AC produced by the \compcert{} front-end without any optimisations.}\label{fig:accumulator_rtl}
    \end{subfigure}
  \end{subfigure}\hfill%
  \begin{subfigure}[b]{0.65\linewidth}
\begin{minted}[fontsize=\tiny,linenos,xleftmargin=20pt]{verilog}
module main(reset, clk, finish, return_val);
  input [0:0] reset, clk;
  output reg [0:0] finish = 0;
  output reg [31:0] return_val = 0;
  reg [31:0] reg_3 = 0, addr = 0, d_in = 0, reg_5 = 0, wr_en = 0;
  reg [0:0] en = 0, u_en = 0;
  reg [31:0] state = 0, reg_2 = 0, reg_4 = 0, d_out = 0, reg_1 = 0;
  reg [31:0] stack [1:0];
  // RAM interface
  always @(negedge clk)
    if ({u_en != en}) begin
      if (wr_en) stack[addr] <= d_in;
      else d_out <= stack[addr];
      en <= u_en;
    end
  // Data-path
  always @(posedge clk)
    case (state)
      32'd11: reg_2 <= d_out;
      32'd8: reg_5 <= 32'd3;
      32'd7: begin u_en <= ( ~ u_en); wr_en <= 32'd1;
                   d_in <= reg_5; addr <= 32'd0; end
      32'd6: reg_4 <= 32'd6;
      32'd5: begin u_en <= ( ~ u_en); wr_en <= 32'd1;
                   d_in <= reg_4; addr <= 32'd1; end
      32'd4: reg_1 <= 32'd1;
      32'd3: reg_3 <= 32'd0;
      32'd2: begin u_en <= ( ~ u_en); wr_en <= 32'd0;
                   addr <= {{{reg_3 + 32'd0} + {reg_1 * 32'd4}} / 32'd4}; end
      32'd1: begin finish = 32'd1; return_val = reg_2; end
      default: ;
    endcase
  // Control logic
  always @(posedge clk)
    if ({reset == 32'd1}) state <= 32'd8;
    else case (state)
           32'd11: state <= 32'd1;        32'd4: state <= 32'd3;
           32'd8: state <= 32'd7;         32'd3: state <= 32'd2;
           32'd7: state <= 32'd6;         32'd2: state <= 32'd11;
           32'd6: state <= 32'd5;         32'd1: ;
           32'd5: state <= 32'd4;         default: ;
         endcase
endmodule
\end{minted}
\caption{Verilog produced by \vericert{}. It demonstrates the instantiation of the RAM (lines 9--15), the data-path (lines 16--32) and the control logic (lines 33--42).}\label{fig:accumulator_v}
\end{subfigure}
\caption{Translating a simple program from C to Verilog.}\label{fig:accumulator_c_rtl}
% \NR{Is the default in line 41 of (c) supposed to be empty?}\YH{Yep, that's how it's generated.}
\end{figure}

\subsection{Translating C to Verilog by Example}
Fig.~\ref{fig:accumulator_c_rtl} illustrates the translation of a simple program that stores and retrieves values from an array.
In this section, we describe the stages of the \vericert{} translation, referring to this program as an example.

\subsubsection{Translating C to 3AC}

The first stage of the translation uses unmodified \compcert{} to transform the C input, shown in Fig.~\ref{fig:accumulator_c}, into a 3AC intermediate representation, shown in Fig.~\ref{fig:accumulator_rtl}.
As part of this translation, function inlining is performed on all functions, which allows us to support function calls without having to support the \texttt{Icall} 3AC instruction.  Although the duplication of the function bodies caused by inlining can increase the area of the hardware, it can have a positive effect on latency and is therefore a common HLS optimisation~\cite{noronha17_rapid_fpga}. Inlining precludes support for recursive function calls, but this feature is not supported in most HLS tools anyway~\cite{davidthomas_asap16}.

%\JW{Is that definitely true? Was discussing this with Nadesh and George recently, and I ended up not being so sure. Inlining could actually lead to \emph{reduced} resource usage because once everything has been inlined, the (big) scheduling problem could then be solved quite optimally. Certainly inlining is known to increase register pressure, but that's not really an issue here. If we're  not sure, we could just say that inlining everything leads to bloated Verilog files and the inability to support recursion, and leave it at that.}\YH{I think that is true, just because we don't do scheduling.  With scheduling I think that's true, inlining actually becomes quite good.}

\subsubsection{Translating 3AC to HTL}

%   + TODO Explain the main mapping in a short simple way

%   + TODO Clarify connection between CFG and FSMD

%   + TODO Explain how memory is mapped
%\JW{I feel like this could use some sort of citation, but I'm not sure what. I guess this is all from "Hardware Design 101", right?}\YH{I think I found a good one actually, which goes over the basics.}
%\JW{I think it would be worth having a sentence to explain how the C model of memory is translated to a hardware-centric model of memory. For instance, in C we have global variables/arrays, stack-allocated variables/arrays, and heap-allocated variables/arrays (anything else?). In Verilog we have registers and RAM blocks. So what's the correspondence between the two worlds? Globals and heap-allocated are not handled, stack-allocated variables become registers, and stack-allocated arrays become RAM blocks? Am I close?}\YH{Stack allocated variables become RAM as well, so that we can deal with addresses easily and take addresses of any variable.} \JW{I see, thanks. So, in short, the only registers in your hardware designs are those that store things like the current state, etc. You generate a fixed number of registers every time you synthesis -- you don't generate extra registers to store any of the program variables. Right?}

The next translation is from 3AC to a new hardware translation language (HTL). %, which is one step towards being completely translated to hardware described in Verilog.
This involves going from a CFG representation of the computation to a finite state machine with data-path (FSMD) representation~\cite{hwang99_fsmd}. The core idea of the FSMD representation is that it separates the control flow from the operations on the memory and registers. %\JP{I've become less comfortable with this term, but it's personal preference so feel free to ignore. I think `generalised finite state machine' (i.e.\ thinking of the entire `data-path' as contributing to the overall state) is more accurate.}\YH{Hmm, yes, I mainly chose FSMD because there is quite a lot of literature around it.  I think for now I'll keep it but for the final draft we could maybe change it.}
%This means that the state transitions can be translated into a simple finite state machine (FSM) where each state contains data operations that update the memory and registers.
Hence, an HTL program consists of two maps from states to Verilog statements: the \emph{control logic} map, which expresses state transitions, and the \emph{data-path} map, which expresses computations.
Fig.~\ref{fig:accumulator_diagram} shows the resulting FSMD architecture. The right-hand block is the control logic that computes the next state, while the left-hand block updates all the registers and RAM based on the current program state.

The HTL language was mainly introduced to simplify the proof of translation from 3AC to Verilog, as these languages have very different semantics.
It serves as an intermediate language with similar semantics to 3AC at the top level, using maps to represents what to execute at every state, and similar semantics to Verilog at the lower level by already using Verilog statements instead of more abstract instructions.
Compared to plain Verilog, HTL is simpler to manipulate and analyse, thereby making it easier to prove optimisations like proper RAM insertion.

\begin{figure*}
  \centering
\definecolor{control}{HTML}{b3e2cd}
\definecolor{data}{HTML}{fdcdac}
\begin{tikzpicture}
  \begin{scope}[scale=1.15]
  \fill[control,fill opacity=1] (6.5,0) rectangle (12,5);
  \fill[data,fill opacity=1] (0,0) rectangle (5.5,5);
  \node at (1,4.7) {Data-path};
  \node at (7.5,4.7) {Control Logic};

  \fill[white,rounded corners=10pt] (7,0.5) rectangle (11.5,2.2);
  \node at (8,2) {\footnotesize Next State FSM};
  \foreach \x in {8,...,2}
    {\pgfmathtruncatemacro{\y}{8-\x}%
      \node[draw,circle,inner sep=0,minimum size=10,scale=0.8] (s\x) at (7.5+\y/2,1.35) {\tiny \x};}
  \node[draw,circle,inner sep=0,minimum size=10,scale=0.8] (s1c) at (11,1.35) {\tiny 1};
  \node[draw,circle,inner sep=0,minimum size=13,scale=0.8] (s1) at (s1c) {};
  \foreach \x in {8,...,3}
    {\pgfmathtruncatemacro{\y}{\x-1}\draw[-{Latex[length=1mm,width=0.7mm]}] (s\x) -- (s\y);}
  \node[draw,circle,inner sep=0,minimum size=10,scale=0.8] (s11) at (10.5,0.9) {\tiny 11};
  \draw[-{Latex[length=1mm,width=0.7mm]}] (s2) -- (s11);
  \draw[-{Latex[length=1mm,width=0.7mm]}] (s11) -- (s1);
  \draw[-{Latex[length=1mm,width=0.7mm]}] (7.2,1.7) to [out=0,in=100] (s8);

  \node[draw,fill=white] (nextstate) at (9.25,3) {\tiny Current State};
  \draw[-{Latex[length=1mm,width=0.7mm]}] let \p1 = (nextstate) in
    (11.5,1.25) -| (11.75,\y1) -- (nextstate);
  \draw let \p1 = (nextstate) in (nextstate) -- (6,\y1) |- (6,1.5);
  \node[scale=0.5,rotate=60] at (7.5,0.75) {\texttt{clk}};
  \node[scale=0.5,rotate=60] at (7.7,0.75) {\texttt{rst}};
  \draw[-{Latex[length=1mm,width=0.7mm]}] (7.65,-0.5) -- (7.65,0.5);
  \draw[-{Latex[length=1mm,width=0.7mm]}] (7.45,-0.5) -- (7.45,0.5);

  \fill[white,rounded corners=10pt] (2,0.5) rectangle (5,3);
  \filldraw[fill=white] (0.25,0.5) rectangle (1.5,2.75);
  \node at (2.6,2.8) {\footnotesize Update};
  \node[align=center] at (0.875,2.55) {\footnotesize \texttt{RAM}};
  \node[scale=0.5] at (4.7,1.5) {\texttt{state}};
  \draw[-{Latex[length=1mm,width=0.7mm]}] (6,1.5) -- (5,1.5);
  \draw[-{Latex[length=1mm,width=0.7mm]}] (6,1.5) -- (7,1.5);
  \node[scale=0.5,rotate=60] at (4.1,0.9) {\texttt{finished}};
  \node[scale=0.5,rotate=60] at (3.9,0.95) {\texttt{return\_val}};
  \node[scale=0.5,rotate=60] at (2.5,0.75) {\texttt{clk}};
  \node[scale=0.5,rotate=60] at (2.7,0.75) {\texttt{rst}};

  \node[scale=0.5,right,inner sep=5pt] (ram1) at (2,2.1) {\texttt{u\_en}};
  \node[scale=0.5,right,inner sep=5pt] (ram2) at (2,1.9) {\texttt{wr\_en}};
  \node[scale=0.5,right,inner sep=5pt] (ram3) at (2,1.7) {\texttt{addr}};
  \node[scale=0.5,right,inner sep=5pt] (ram4) at (2,1.5) {\texttt{d\_in}};
  \node[scale=0.5,right,inner sep=5pt] (ram5) at (2,1.3) {\texttt{d\_out}};

  \node[scale=0.5,left,inner sep=5pt] (r1) at (1.5,2.1) {\texttt{u\_en}};
  \node[scale=0.5,left,inner sep=5pt] (r2) at (1.5,1.9) {\texttt{wr\_en}};
  \node[scale=0.5,left,inner sep=5pt] (r3) at (1.5,1.7) {\texttt{addr}};
  \node[scale=0.5,left,inner sep=5pt] (r4) at (1.5,1.5) {\texttt{d\_in}};
  \node[scale=0.5,left,inner sep=5pt] (r5) at (1.5,1.3) {\texttt{d\_out}};

  \draw[-{Latex[length=1mm,width=0.7mm]}] (ram1) -- (r1);
  \draw[-{Latex[length=1mm,width=0.7mm]}] (ram2) -- (r2);
  \draw[-{Latex[length=1mm,width=0.7mm]}] (ram3) -- (r3);
  \draw[-{Latex[length=1mm,width=0.7mm]}] (ram4) -- (r4);
  \draw[-{Latex[length=1mm,width=0.7mm]}] (r5) -- (ram5);

  \draw[-{Latex[length=1mm,width=0.7mm]}] (4,0.5) -- (4,-0.5);
  \draw[-{Latex[length=1mm,width=0.7mm]}] (3.75,0.5) -- (3.75,-0.5);
  \draw[-{Latex[length=1mm,width=0.7mm]}] (2.45,-0.5) -- (2.45,0.5);
  \draw[-{Latex[length=1mm,width=0.7mm]}] (2.65,-0.5) -- (2.65,0.5);

  \foreach \x in {0,...,1}
  {\draw (0.25,1-0.25*\x) -- (1.5,1-0.25*\x); \node at (0.875,0.88-0.25*\x) {\tiny \x};}

  %\node[scale=0.5] at (1.2,2.2) {\texttt{wr\_en}};
  %\node[scale=0.5] at (1.2,2) {\texttt{wr\_addr}};
  %\node[scale=0.5] at (1.2,1.8) {\texttt{wr\_data}};
  %\node[scale=0.5] at (1.2,1.4) {\texttt{r\_addr}};
  %\node[scale=0.5] at (1.2,1.2) {\texttt{r\_data}};
  %
  %\node[scale=0.5] at (2.3,2.2) {\texttt{wr\_en}};
  %\node[scale=0.5] at (2.3,2) {\texttt{wr\_addr}};
  %\node[scale=0.5] at (2.3,1.8) {\texttt{wr\_data}};
  %\node[scale=0.5] at (2.3,1.4) {\texttt{r\_addr}};
  %\node[scale=0.5] at (2.3,1.2) {\texttt{r\_data}};
  %
  %\draw[-{Latex[length=1mm,width=0.7mm]}] (2,2.2) -- (1.5,2.2);
  %\draw[-{Latex[length=1mm,width=0.7mm]}] (2,2) -- (1.5,2);
  %\draw[-{Latex[length=1mm,width=0.7mm]}] (2,1.8) -- (1.5,1.8);
  %\draw[-{Latex[length=1mm,width=0.7mm]}] (2,1.4) -- (1.5,1.4);
  %\draw[-{Latex[length=1mm,width=0.7mm]}] (1.5,1.2) -- (2,1.2);

  \filldraw[fill=white] (2.8,3.25) rectangle (4.2,4.75);
  \node at (3.5,4.55) {\footnotesize \texttt{Registers}};
  \draw[-{Latex[length=1mm,width=0.7mm]}] (2,2.4) -| (1.75,4) -- (2.8,4);
  \draw[-{Latex[length=1mm,width=0.7mm]}] (4.2,4) -- (5.25,4) |- (5,2.4);
  \draw[-{Latex[length=1mm,width=0.7mm]}] (5.25,2.4) -- (6.2,2.4) |- (7,1.8);

  \node[scale=0.5] at (3.5,4.2) {\texttt{reg\_1}};
  \node[scale=0.5] at (3.5,4) {\texttt{reg\_2}};
  \node[scale=0.5] at (3.5,3.8) {\texttt{reg\_3}};
  \node[scale=0.5] at (3.5,3.6) {\texttt{reg\_4}};
  \node[scale=0.5] at (3.5,3.4) {\texttt{reg\_5}};
\end{scope}
\end{tikzpicture}
%  \alt{Diagram displaying the data-path and its internal modules, as well as the control logic and its state machine.}
  \caption{The FSMD for the example shown in Fig.~\ref{fig:accumulator_c_rtl}, split into a data-path and control logic for the next state calculation.  The Update block takes the current state, current values of all registers and at most one value stored in the RAM, and calculates a new value that can either be stored back in the RAM or in a register.}\label{fig:accumulator_diagram}
\end{figure*}

%\JP{Does it? Verilog has neither physical registers nor RAMs, just language constructs which the synthesiser might implement with registers and RAMs. We should be clear whether we're talking about the HDL representation, or the synthesised result: in our case these can be very different since we don't target any specific architectural features of an FPGA fabric of ASIC process.}
\paragraph{Translating memory}
Typically, HLS-generated hardware consists of a sea of registers and RAMs.
This memory view is very different from the C memory model, so we perform the following translation from \compcert{}'s abstract memory model to a concrete RAM.\@
Variables that do not have their address taken are kept in registers, which correspond to the registers in 3AC.
All address-taken variables, arrays, and structs are kept in RAM.
The stack of the main function becomes an unpacked array of 32-bit integers representing the RAM block.  Any loads and stores are temporarily translated to direct accesses to this array, where each address has its offset removed and is divided by four.  In a separate HTL-to-HTL conversion, these direct accesses are then translated to proper loads and stores that use a RAM interface to communicate with the RAM, shown on lines 21, 24 and 28 of Fig.~\ref{fig:accumulator_v}.  This pass inserts a RAM block with the interface around the unpacked array.  Without this interface and without the RAM block, the synthesis tool processing the Verilog hardware description would not identify the array as a RAM, and would instead implement it using many registers.  This interface is shown on lines 9--15 in the Verilog code in Fig.~\ref{fig:accumulator_v}.
A high-level overview of the architecture and of the RAM interface can be seen in Fig.~\ref{fig:accumulator_diagram}.

\paragraph{Translating instructions}

Most 3AC instructions correspond to hardware constructs.
%Each 3AC instruction either corresponds to a hardware construct or does not have to be handled by the translation, such as function calls (because of inlining). \JW{Are function calls the only 3AC instruction that we ignore? (And I guess return statements too for the same reason.)}\YH{Actually, return instructions are translated (because you can return from main whenever), so call instructions (Icall, Ibuiltin and Itailcall) are the only functions that are not handled.}
% JW: Thanks; please check proposed new text.
For example, line 2 in Fig.~\ref{fig:accumulator_rtl} shows a 32-bit register \texttt{x5} being initialised to 3, after which the control flow moves execution to line 3. This initialisation is also encoded in the Verilog generated from HTL at state 8 in both the control logic and data-path always-blocks, shown at lines 33 and 16 respectively in Fig.~\ref{fig:accumulator_v}.  Simple operator instructions are translated in a similar way.  For example, the add instruction is just translated to the built-in add operator, similarly for the multiply operator.  All 32-bit instructions can be translated in this way, but some special instructions require extra care. One such instruction is the \texttt{Oshrximm} instruction, which is discussed further in Section~\ref{sec:algorithm:optimisation:oshrximm}. Another is the \texttt{Oshldimm} instruction, which is a left rotate instruction that has no Verilog equivalent and therefore has to be implemented in terms of other operations and proven to be equivalent.
% In addition to any non-32-bit operations, the remaining
The only 32-bit instructions that we do not translate are case-statements (\texttt{Ijumptable}) and those instructions related to function calls (\texttt{Icall}, \texttt{Ibuiltin}, and \texttt{Itailcall}), because we enable inlining by default.

\subsubsection{Translating HTL to Verilog}

Finally, we have to translate the HTL code into proper Verilog. % and prove that it behaves the same as the 3AC according to the Verilog semantics.
The challenge here is to translate our FSMD representation into a Verilog AST.  However, as all the instructions in HTL are already expressed as Verilog statements, only the top-level data-path and control logic maps need to be translated to valid Verilog case-statements.  We also require declarations for all the variables in the program, as well as declarations of the inputs and outputs to the module, so that the module can be used inside a larger hardware design.  In addition to translating the maps of Verilog statements, an always-block that will behave like the RAM also has to be created, which is only modelled abstractly at the HTL level.
Fig.~\ref{fig:accumulator_v} shows the final Verilog output that is generated for our example.

Although this translation seems quite straight\-forward, proving that this translation is correct is complex.
All the implicit assumptions that were made in HTL need to be translated explicitly to Verilog statements and it needs to be shown that these explicit behaviours are equivalent to the assumptions made in the HTL semantics.  One main example of this is proving that the specification of the RAM in HTL does indeed behave in the same as its Verilog implementation.
We discuss these proofs in upcoming sections.

 %In general, the generated Verilog structure has similar to that of the HTL code.
%The key difference is that the control and datapath maps become Verilog case-statements.
%Other additions are the initialisation of all the variables in the code to the
%correct bitwidths and the declaration of the inputs and outputs to the module,
%so that the module can be used inside a larger hardware design.

\section{Formulating the Correctness Theorem}

The main correctness theorem is analogous to that stated in
\compcert{}~\cite{leroy09_formal_verif_realis_compil}: for all Clight source
programs $C$, if the translation to the target Verilog code succeeds,
$\mathit{Safe}(C)$ holds and the target Verilog has behaviour $B$ when
simulated, then $C$ will have the same behaviour $B$. $\mathit{Safe}(C)$ means
all observable behaviours of $C$ are safe, which can be defined as
$\forall B,\ C \Downarrow B \implies B \in \texttt{Safe}$.  A behaviour is in
\texttt{Safe} if it is either a final state (in the case of convergence) or
divergent, but it cannot `go wrong'. (This means that the source program must
not contain undefined behaviour.) In \compcert{}, a behaviour is also associated
with a trace of I/O events, but since external function calls are not supported
in \vericert{}, this trace will always be empty.

\begin{theorem}
  Whenever the translation from $C$ succeeds and produces Verilog $V$, and all
  observable behaviours of $C$ are safe, then $V$ has behaviour $B$ only if $C$
  has behaviour $B$.
  \begin{equation*}
    \forall C, V, B,\quad \yhfunction{HLS} (C) = \yhconstant{OK} (V) \land \mathit{Safe}(C) \implies (V \Downarrow B \implies C \Downarrow B).
  \end{equation*}
\end{theorem}

Why is this correctness theorem also the right one for HLS? It could be argued
that hardware inherently runs forever and therefore does not produce a
definitive final result.  This would mean that the \compcert{} correctness
theorem would probably be unhelpful with proving hardware correctness, as the
behaviour would always be divergent.  However, in practice, HLS does not
normally produce the top-level of the design that needs to connect to other
components, therefore needing to run forever.  Rather, HLS often produces
smaller components that take an input, execute, and then terminate with an
answer.  To start the execution of the hardware and to signal to the HLS
component that the inputs are ready, the $\mathit{rst}$ signal is set and unset.
Then, once the result is ready, the $\mathit{fin}$ signal is set and the result
value is placed in $\mathit{ret}$.  These signals are also present in the
semantics of execution shown in Fig.~\ref{fig:inference_module}.  The
correctness theorem therefore also uses these signals, and the proof shows that
once the $\mathit{fin}$ flag is set, the value in $\mathit{ret}$ is correct
according to the semantics of Verilog and Clight.  Note that the compiler is
allowed to fail and not produce any output; the correctness theorem only applies
when the translation succeeds.

How can we prove this theorem? First, note that the theorem is a `backwards
simulation' result (every target behaviour must also be a source behaviour),
following the terminology used in the \compcert{}
literature~\cite{leroy09_formal_verif_realis_compil}. The reverse direction
(every source behaviour must also be a target behaviour) is not demanded because
compilers are permitted to resolve any non-determinism present in their source
programs. However, since Clight programs are all deterministic, as are the
Verilog programs in the fragment we consider, we can actually reformulate the
correctness theorem above as a forwards simulation result (following standard
\compcert{} practice), which makes it easier to prove.  To prove this forward
simulation, it suffices to prove forward simulations between each pair of
consecutive intermediate languages, as these results can be composed to prove
the correctness of the whole HLS tool.  The forward simulation from 3AC to HTL
is stated in Lemma~\ref{lemma:htl} (Section~\ref{sec:proof:3ac_htl}), the
forward simulation for the RAM insertion is shown in Lemma~\ref{lemma:htl_ram}
(Section~\ref{sec:proof:ram_insertion}), then the forward simulation between HTL
and Verilog is shown in Lemma~\ref{lemma:verilog}
(Section~\ref{sec:proof:htl_verilog}), and finally, the proof that Verilog is
deterministic is given in Lemma~\ref{lemma:deterministic}
(Section~\ref{sec:proof:deterministic}).

\section{A Formal Semantics for Verilog}\label{sec:verilog}

\newcommand{\alwaysblock}{always-block}

This section describes the Verilog semantics that was chosen for the target language, including the changes that were made to the semantics to make it a suitable HLS target.  The Verilog standard is quite large~\cite{06_ieee_stand_veril_hardw_descr_languag,05_ieee_stand_veril_regis_trans_level_synth}, but the syntax and semantics can be reduced to a small subset that \vericert{} needs to target.  This section  also describes how \vericert{}'s representation of memory differs from \compcert{}'s memory model.

The Verilog semantics we use is ported to Coq from a semantics written in HOL4 by \textcite{loow19_proof_trans_veril_devel_hol} and used to prove the translation from HOL4 to Verilog~\cite{loow19_verif_compil_verif_proces}. % which was used to create a formal translation from a logic representation encoded in the HOL4~\cite{slind08_brief_overv_hol4} theorem prover into an equivalent Verilog design.
This semantics is quite practical as it is restricted to a small subset of Verilog, which can nonetheless be used to model the hardware constructs required for HLS.  The main features that are excluded are continuous assignment and combinational \alwaysblock{}s; these are modelled in other semantics such as that by~\textcite{meredith10_veril}. %however, these are not necessarily needed, but require more complicated event queues and execution model.

The semantics of Verilog differs from regular programming languages, as it is used to describe hardware directly, which is inherently parallel, rather than an algorithm, which is usually sequential.  The main construct in Verilog is the \alwaysblock{}.
A module can contain multiple \alwaysblock{}s, all of which run in parallel.  These \alwaysblock{}s further contain statements such as if-statements or assignments to variables.  We support only \emph{synchronous} logic, which means that the \alwaysblock{} is triggered on (and only on) the positive or negative edge of a clock signal.
%\NR{We should mention that variables cannot be driven by multiple \alwaysblock{}s, since one might get confused with data races when relating to concurrent processes in software.} \JW{Given the recent discussion on Teams, it seems to me that we perhaps don't need to mention here what happens if a variable is driven multiple times per clock cycle, especially since \vericert{} isn't ever going to do that.}

The semantics combines the big-step and small-step styles. The overall execution of the hardware is described using a small-step semantics, with one small step per clock cycle; this is appropriate because hardware is routinely designed to run for an unlimited number of clock cycles and the big-step style is ill-suited to describing infinite executions. Then, within each clock cycle, a big-step semantics is used to execute all the statements.
An example of a rule for executing an \alwaysblock{} that is triggered at the positive edge of the clock is shown below, where $\Sigma$ is the state of the registers in the module and $s$ is the statement inside the \alwaysblock{}:

\begin{equation*}
  \inferrule[Always]{(\Sigma, s)\downarrow_{\text{stmnt}} \Sigma'}{(\Sigma, \yhkeyword{always @(posedge clk) } s) \downarrow_{\text{always}^{+}} \Sigma'}
\end{equation*}

\noindent This rule says that assuming the statement $s$ in the \alwaysblock{} runs with state $\Sigma$ and produces the new state $\Sigma'$, the \alwaysblock{} will result in the same final state.  %Since only clocked \alwaysblock{} are supported, and one step in the semantics correspond to one clock cycle, it means that this rule is run once per clock cycle for each \alwaysblock{}.

Two types of assignments are supported in \alwaysblock{}s: nonblocking and blocking assignment.  Nonblocking assignments all take effect simultaneously at the end of the clock cycle, %and atomically.
while blocking assignments happen instantly so that later assignments in the clock cycle can pick them up.  To model both of these assignments, the state $\Sigma$ has to be split into two maps: $\Gamma$, which contains the current values of all variables and arrays, and $\Delta$, which contains the values that will be assigned at the end of the clock cycle. $\Sigma$ can therefore be defined as follows: $\Sigma = (\Gamma, \Delta)$.
Nonblocking assignment can therefore be expressed as follows:
\begin{equation*}
  \inferrule[Nonblocking Reg]{\yhkeyword{name}\ d = \yhkeyword{OK}\ n \\ (\Gamma, e) \downarrow_{\text{expr}} v}{((\Gamma, \Delta), d\ \yhkeyword{ <= } e) \downarrow_{\text{stmnt}} (\Gamma, \Delta [n \mapsto v])}\\
\end{equation*}

\noindent where assuming that $\downarrow_{\text{expr}}$ evaluates an expression $e$ to a value $v$, the nonblocking assignment $d\ \yhkeyword{ <= } e$ updates the future state of the variable $d$ with value $v$.

Finally, the following rule dictates how the whole module runs in one clock cycle:
\begin{equation*}
  \inferrule[Module]{(\Gamma, \epsilon, \vec{m})\ \downarrow_{\text{module}} (\Gamma', \Delta')}{(\Gamma, \yhkeyword{module } \yhconstant{main} \yhkeyword{(...);}\ \vec{m}\ \yhkeyword{endmodule}) \downarrow_{\text{program}} (\Gamma'\ //\ \Delta')}
\end{equation*}
where $\Gamma$ is the initial state of all the variables, $\epsilon$ is the empty map because the $\Delta$ map is assumed to be empty at the start of the clock cycle, and $\vec{m}$ is a list of variable declarations and \alwaysblock{}s that $\downarrow_{\text{module}}$ evaluates sequentially to obtain $(\Gamma', \Delta')$. The final state is obtained by merging these maps using the $//$ operator, which gives priority to the right-hand operand in a conflict. This rule ensures that the nonblocking assignments overwrite at the end of the clock cycle any blocking assignments made during the cycle.

\subsection{Changes to the Semantics}

Five changes were made to the semantics proposed by \textcite{loow19_proof_trans_veril_devel_hol} to make it suitable as an HLS target.

\paragraph{Adding array support}
The main change is the addition of support for arrays, which are needed to model RAM in Verilog.  RAM is needed to model the stack in C efficiently, without having to declare a variable for each possible stack location. % In the original semantics, RAMs (as well as inputs and outputs to the module) could be modelled using a function from variable names (strings) to values, which could be modified accordingly to model inputs to the module.  This is quite an abstract description of memory and can also be expressed as an array of bitvectors instead, which is the path we took. This requires the addition of array operators to the semantics and correct reasoning of loads and stores to the array in different \alwaysblock{}s simultaneously.
Consider the following Verilog code:

\begin{center}
\begin{minted}[xleftmargin=40pt,linenos]{verilog}
reg [31:0] x[1:0];
always @(posedge clk) begin x[0] = 1; x[1] <= 1; end
\end{minted}
\end{center}

which modifies one array element using blocking assignment and then a second using nonblocking assignment. If the existing semantics were used to update the array, then during the merge, the entire array \texttt{x} from the nonblocking association map would replace the entire array from the blocking association map.  This would replace \texttt{x[0]} with its original value and therefore behave incorrectly. Accordingly, we modified the maps so they record updates on a per-el\-em\-ent basis. Our state $\Gamma$ is therefore further split up into $\Gamma_{r}$ for instantaneous updates to variables, and $\Gamma_{a}$ for instantaneous updates to arrays ($\Gamma = (\Gamma_{r}, \Gamma_{a})$); $\Delta$ is split similarly ($\Delta = (\Delta_{r}, \Delta_{a})$). The merge function then ensures that only the modified indices get updated when $\Gamma_{a}$ is merged with the nonblocking map equivalent $\Delta_{a}$.

\paragraph{Adding negative edge support}
To reason about circuits that execute on the negative edge of the clock (such as our RAM interface described in Section~\ref{sec:algorithm:optimisation:ram}),  support for negative-edge-triggered \alwaysblock{}s was added to the semantics. This is shown in the modifications of the \textsc{Module} rule shown below:

\begin{equation*}
  \inferrule[Module]{(\Gamma, \epsilon, \vec{m})\ \downarrow_{\text{module}^{+}} (\Gamma', \Delta') \\ (\Gamma'\ //\ \Delta', \epsilon, \vec{m}) \downarrow_{\text{module}^{-}} (\Gamma'', \Delta'')}{(\Gamma, \yhkeyword{module}\ \yhconstant{main} \yhkeyword{(...);}\ \vec{m}\ \yhkeyword{endmodule}) \downarrow_{\text{program}} (\Gamma''\ //\ \Delta'')}
\end{equation*}

The main execution of the module $\downarrow_{\text{module}}$ is split into $\downarrow_{\text{module}^{+}}$ and $\downarrow_{\text{module}^{-}}$, which are rules that only execute \alwaysblock{}s triggered at the positive and at the negative edge respectively. The positive-edge-triggered \alwaysblock{}s are processed in the same way as in the original \textsc{Module} rule. The output maps $\Gamma'$ and $\Delta'$ are then merged and passed as the blocking assignments map into the negative edge execution, so that all the blocking and nonblocking assignments are present.  Finally, all the negative-edge-triggered \alwaysblock{}s are processed and merged to give the final state.

\paragraph{Adding declarations} Explicit support for declaring inputs, outputs and internal variables was added to the semantics to make sure that the generated Verilog also contains the correct declarations.  This adds some guarantees to the generated Verilog and ensures that it synthesises and simulates correctly.

\paragraph{Removing support for external inputs to modules} Support for receiving external inputs was removed from the semantics for simplicity, as these are not needed for an HLS target. The main module in Verilog models the main function in C, and since the inputs to a C function should not change during its execution, there is no need for external inputs for Verilog modules.

\paragraph{Simplifying representation of bitvectors} Finally, we use 32-bit integers to represent bitvectors rather than arrays of booleans. This is because \vericert{} (currently) only supports types represented by 32 bits.

\subsection{Integrating the Verilog Semantics into \compcert{}'s Model}
\label{sec:verilog:integrating}

\begin{figure*}
  \centering
  \begin{minipage}{1.0\linewidth}
    \begin{gather*}
      \inferrule[Step]{\Gamma_r[\mathit{rst}] = 0 \\ \Gamma_r[\mathit{fin}] = 0 \\ (m, (\Gamma_r, \Gamma_a))\ \downarrow_{\text{program}} (\Gamma_r', \Gamma_a')}{\yhconstant{State}\ \mathit{sf}\ m\ \ \Gamma_r[\sigma]\ \ \Gamma_r\ \Gamma_a \longrightarrow \yhconstant{State}\ \mathit{sf}\ m\ \ \Gamma_r'[\sigma]\ \ \Gamma_r'\ \Gamma_a'}\\
      %
      \inferrule[Finish]{\Gamma_r[\mathit{fin}] = 1}{\yhconstant{State}\ \mathit{sf}\ m\ \sigma\ \Gamma_r\ \Gamma_a \longrightarrow \yhconstant{Returnstate}\ \mathit{sf}\ \Gamma_r[ \mathit{ret}]}\\
      %
      \inferrule[Call]{ }{\yhconstant{Callstate}\ \mathit{sf}\ m\ \vec{r} \longrightarrow \yhconstant{State}\ \mathit{sf}\ m\ n\ ((\yhfunction{init\_params}\ \vec{r}\ a)[\sigma \mapsto n, \mathit{fin} \mapsto 0, \mathit{rst} \mapsto 0])\ \epsilon}\\
      %
      \inferrule[Return]{ }{\yhconstant{Returnstate}\ (\yhconstant{Stackframe}\ r\ m\ \mathit{pc}\ \Gamma_r\ \Gamma_a :: \mathit{sf})\ v \longrightarrow \yhconstant{State}\ \mathit{sf}\ m\ \mathit{pc}\ (\Gamma_{r} [ \sigma \mapsto \mathit{pc}, r \mapsto v ])\ \Gamma_{a}}
    \end{gather*}
  \end{minipage}
  \caption{Top-level small-step semantics for Verilog modules in \compcert{}'s computational framework.}%
  \label{fig:inference_module}
\end{figure*}

The \compcert{} computation model defines a set of states through which execution passes. In this subsection, we explain how we extend our Verilog semantics with four special-purpose  registers in order to integrate it into \compcert{}.

\compcert{} executions pass through three main states:
\begin{description}
  \item[\texttt{State} $\mathit{sf}$ $m$ $v$ $\Gamma_{r}$ $\Gamma_{a}$] The main state when executing a function, with stack frame $\mathit{sf}$, current module $m$, current state $v$ and variable states $\Gamma_{r}$ and $\Gamma_{a}$.
  \item[\texttt{Callstate} $\mathit{sf}$ $m$ $\vec{r}$] The state that is reached when a function is called, with the current stack frame $\mathit{sf}$, current module $m$ and arguments $\vec{r}$.
  \item[\texttt{Returnstate} $\mathit{sf}$ $v$] The state that is reached when a function returns back to the caller, with stack frame $\mathit{sf}$ and return value $v$.
\end{description}

To support this computational model, we extend the Verilog module we generate with the following four registers and a RAM block:

\begin{description}
  \item[program counter] The program counter can be modelled using a register that keeps track of the state, denoted as $\sigma$.
  \item[function entry point] When a function is called, the entry point denotes the first instruction that will be executed. This can be modelled using a reset signal that sets the state accordingly, denoted as $\mathit{rst}$.
  \item[return value] The return value can be modelled by setting a finished flag to 1 when the result is ready, and putting the result into a 32-bit output register. These are denoted as $\mathit{fin}$ and $\mathit{ret}$ respectively.
%\JW{Is there a mismatch between `ret' in the figure and `rtrn' in the text?}
  \item[stack] The function stack can be modelled as a RAM block, which is implemented using an array in the module, and denoted as $\mathit{stk}$.
%\JW{Is there a mismatch between `st' in the figure and `stk' in the text?}\YH{It was actually between $\Gamma_{a}$ and \mathit{stk}.  The \mathit{st} should have been $\sigma$.}
\end{description}

Fig.~\ref{fig:inference_module} shows the inference rules for moving between the computational states.  The first, \textsc{Step}, is the normal rule of execution.  It defines one step in the \texttt{State} state, assuming that the module is not being reset, that the finish state has not been reached yet, that the current and next state are $v$ and $v'$, and that the module runs from state $\Gamma$ to $\Gamma'$ using the \textsc{Step} rule.  The \textsc{Finish} rule returns the final value of running the module and is applied when the $\mathit{fin}$ register is set; the return value is then taken from the $\mathit{ret}$ register.

Note that there is no step from \texttt{State} to \texttt{Callstate}; this is because function calls are not supported, and it is therefore impossible in our semantics ever to reach a \texttt{Callstate} except for the initial call to main. So the \textsc{Call} rule is only used at the very beginning of execution; likewise, the \textsc{Return} rule is only matched for the final return value from the main function.
Therefore, in addition to the rules shown in Fig.~\ref{fig:inference_module}, an initial state and final state need to be defined:

\begin{gather*}
  \inferrule[Initial]{\yhfunction{is\_internal}\ P.\texttt{main}}{\yhfunction{initial\_state}\ (\yhconstant{Callstate}\ []\ P.\texttt{main}\ [])}\qquad
  \inferrule[Final]{ }{\yhfunction{final\_state}\ (\yhconstant{Returnstate}\ []\ n)\ n}
\end{gather*}

\noindent where the initial state is the \texttt{Callstate} with an empty stack frame and no arguments for the \texttt{main} function of program $P$, where this \texttt{main} function needs to be in the current translation unit.  The final state results in the program output of value $n$ when reaching a \texttt{Returnstate} with an empty stack frame.

\subsection{Memory Model}\label{sec:verilog:memory}

The Verilog semantics do not define a memory model for Verilog, as this is not needed for a hardware description language.  There is no preexisting architecture that Verilog will produce; it can describe any memory layout that is needed.  Instead of having specific semantics for memory, the semantics only needs to support the language features that can produce these different memory layouts, these being Verilog arrays.  We therefore define semantics for updating Verilog arrays using blocking and nonblocking assignment.  We then have to prove that the C memory model that \compcert{} uses matches with the interpretation of arrays used in Verilog.  The \compcert{} memory model is infinite, whereas our representation of arrays in Verilog is inherently finite.  There have already been efforts to define a general finite memory model for all intermediate languages in \compcert{}, such as \compcert{}\-S~\cite{besson18_compc} or \compcert{}-TSO~\cite{sevcik13_compc}, or keeping the intermediate languages intact and translate to a more concrete finite memory model in the back end, such as in \compcert{}\-ELF~\cite{wang20_compc}.  We also define such a translation from \compcert{}'s standard infinite memory model to finite arrays that can be represented in Verilog.  There is therefore no more notion of an abstract memory model and all the interactions to memory are encoded in the hardware itself.

%\JW{I'm not quite sure I understand. Let me check: Are you saying that previous work has shown how all the existing CompCert passes can be adapted from an infinite to a finite memory model, but what we're doing is leaving the default (infinite) memory model for the CompCert front end, and just converting from an infinite memory model to a finite memory model when we go from 3AC to HTL?}\YH{Yes exactly, most papers changed the whole memory model to thread through properties that were then needed in the back end, but we currently don't need to do that.  I need to double check though for CompCertELF, it doesn't actually seem to be the case.  Will edit this section later.}

\definecolor{compcertmemmodel}{HTML}{e2ccea}
\definecolor{vericertmemmodel}{HTML}{cbe1db}
\begin{figure}
  \centering
  \begin{tikzpicture}
    \fill[compcertmemmodel,rounded corners=3pt] (0,0) rectangle (5,-5);
    \fill[vericertmemmodel,rounded corners=3pt] (7,0) rectangle (12,-5);
    \node[right] at (0,-0.3) {\small \textbf{\compcert{}'s Memory Model}};
    \node[right] at (7,-0.3) {\small \textbf{Verilog Memory Representation}};
    \node[right] (x0) at (0.2,-1.9) {\small 0};
    \node[right] (x1) at (0.2,-2.5) {\small 1};
    \node[rotate=90] (x2) at (0.43,-3.1) {$\cdots$};
    \foreach \x in {0,...,6}{%
      \node[right] (s\x) at (2.5,-1-\x*0.3) {\small \x};
      \node[right] (t\x) at (4,-1-\x*0.3) {};
      \draw[->] (s\x) -- (t\x);
    }

    \node[right] at (t0) {\small \texttt{DE}};
    \node[right] at (t1) {\small \texttt{AD}};
    \node[right] at (t2) {\small \texttt{BE}};
    \node[right] at (t3) {\small \texttt{EF}};
    \node[right] at (t4) {\small \texttt{12}};
    \node[right] at (t5) {\small \texttt{34}};
    \node[right] at (t6) {\small \texttt{56}};
    \node[right] at (3.1,-3.1) {$\cdots$};

    \node[right] at (3.1,-4) {$\cdots$};
    \node[scale=1.3] at (6,-2.5) {\Huge $\Rightarrow$};

    \draw[->] (x0) -- (s3);
    \draw[->] (x1) -- (2.5,-4);
    \draw (0,-4.3) -- (5,-4.3);
    \node at (2.5,-4.7) {\small \texttt{x[0] = 0xDEADBEEF;}};

    \draw (7.2,-1.2) rectangle (9.4,-3.9);
    \draw (9.6,-1.2) rectangle (11.8,-3.9);

    \foreach \x in {0,...,8}{%
      \draw (7.2,-1.2-\x*0.3) -- (9.4,-1.2-\x*0.3);
      \draw (9.6,-1.2-\x*0.3) -- (11.8,-1.2-\x*0.3);
      \node (b\x) at (8.3,-1.35-\x*0.3) {};
      \node (nb\x) at (10.7,-1.35-\x*0.3) {};
    }

    \node[scale=1.2] at (b0) {\tiny\texttt{0: Some 00000000}};
    \node[scale=1.2] at (b1) {\tiny\texttt{1: Some 12345600}};
    \node[scale=1.2] at (b2) {\tiny\texttt{2: Some 00000000}};
    \node[scale=1.2] at (b3) {\tiny\texttt{3: Some 00000000}};
    \node[scale=1.2] at (b4) {\tiny\texttt{4: Some 00000000}};
    \node[scale=1.2] at (b5) {\tiny\texttt{5: Some 00000000}};
    \node[scale=1.2] at (b6) {\tiny\texttt{6: Some 00000000}};
    \node[scale=1.2] at ($(b7) - (0,0.05)$) {$\cdots$};
    \node[scale=1.2] at (b8) {\tiny\texttt{N: Some 00000000}};

    \node[scale=1.2] at (nb0) {\tiny\texttt{0: Some DEADBEEF}};
    \node[left,scale=1.2] at (nb1) {\tiny\texttt{1: None}};
    \node[left,scale=1.2] at (nb2) {\tiny\texttt{2: None}};
    \node[left,scale=1.2] at (nb3) {\tiny\texttt{3: None}};
    \node[left,scale=1.2] at (nb4) {\tiny\texttt{4: None}};
    \node[left,scale=1.2] at (nb5) {\tiny\texttt{5: None}};
    \node[left,scale=1.2] at (nb6) {\tiny\texttt{6: None}};
    \node[scale=1.2] at ($(nb7) - (0,0.05)$) {$\cdots$};
    \node[left,scale=1.2] at (nb8) {\tiny\texttt{N: None}};

    \node at (8.3,-1) {$\Gamma_{a}$};
    \node at (10.7,-1) {$\Delta_{a}$};

    \draw (7,-4.3) -- (12,-4.3);
    \node at (9.5,-4.7) {\small \texttt{stack[0] <= 0xDEADBEEF;}};
  \end{tikzpicture}
  %\Description{\compcert{}'s memory model is translated into a more concrete memory model based on Verilog arrays.  Two association maps are therefore needed to keep track of the blocking and nonblocking assignments.}
  \caption{Change in the memory model during the translation of 3AC into HTL.  The state of the memories in each case is right after the execution of the store to memory.}\label{fig:memory_model_transl}
\end{figure}

%\JW{It's not completely clear what the relationship is between your work and those works. The use of `only' suggests that you've re-done a subset of work that has already been done -- is that the right impression?}\YH{Hopefully that's more clear.}

This translation is represented in Fig.~\ref{fig:memory_model_transl}.  \compcert{} defines a map from blocks to maps from memory addresses to memory contents.  Each block represents an area in memory; for example, a block can represent a global variable or a stack for a function. As there are no global variables, the main stack can be assumed to be block 0, and this is the only block we translate.
%\JW{So the stack frame for a function called by main would be in a different block, is that the idea? Seems unusual not to have a single stack.}
%\YH{Yeah exactly, it makes it much easier to reason about though, because everything is nicely isolated.  This is exactly what CompCertELF and CompCertS try and solve though.}
%\JW{Would global variables normally be put in blocks 1, 2, etc.?}
%\YH{Yes, although it may also be possible that they could be numbered 0, 1, 2, 3, 4, pushing the block of the stack higher.}
Meanwhile, our Verilog semantics defines two finite arrays of optional values, one for the blocking assignments map $\Gamma_{\mathrm{a}}$ and one for the nonblocking assignments map $\Delta_{\mathrm{a}}$.
%\JW{It's a slight shame that `block' is used in two different senses in the preceding two sentences. I guess that can't be helped.}
%\YH{Ah that's true, I hadn't even noticed.  Yeah I think it would be good to keep the name ``block'' for CompCert's blocks.}
The optional values are present to ensure correct merging of the two association maps at the end of the clock cycle. %During our translation we only convert block 0 to a Verilog memory, and ensure that it is the only block that is present.
%This means that the block necessarily represents the stack of the main function.
The invariant used in the proofs is that block 0 should be equivalent to the merged representation of the $\Gamma_{\mathrm{a}}$ and $\Delta_{\mathrm{a}}$ maps.

%However, in practice, assigning and reading from an array directly in the state machine will not produce a memory in the final hardware design, as the synthesis tool cannot identify the array as having the necessary properties that a RAM needs, even though this is the most natural formulation of memory.  Even though theoretically the memory will only be read from once per clock cycle, the synthesis tool cannot ensure that this is true, and will instead create a register for each memory location.  This increases the size of the circuit dramatically, as the RAM on the FPGA chip will not be reused.  Instead, the synthesis tool expects a specific interface that ensures these properties, and will then transform the interface into a proper RAM during synthesis.  Therefore, a translation has to be performed from the naive use of memory in the state machine, to a proper use of a memory interface.

%\begin{figure}
%  \centering
%  \begin{subfigure}[t]{0.48\linewidth}
%    \includegraphics[width=\linewidth]{diagrams/store_waveform.pdf}
%    \caption{Store waveform.}
%  \end{subfigure}\hfill%
%  \begin{subfigure}[t]{0.48\linewidth}
%    \includegraphics[width=\linewidth]{diagrams/load_waveform.pdf}
%    \caption{Load waveform.}
%  \end{subfigure}
%\end{figure}

\subsection{Deterministic Verilog Semantics}%
\label{sec:proof:deterministic}

% Finally, to obtain the backward simulation that we want, it has to be shown
% that if we generate hardware with a specific behaviour, that it is the only
% possible program with that behaviour.  This only has to be performed for the
% final intermediate language, which is Verilog, so that the backward simulation
% holds for the whole chain from Clight to Verilog.
The final lemma we need is that the Verilog semantics is deterministic. This
result allows us to replace the forwards simulation we have proved with the
backwards simulation we desire.

\begin{lemma}\label{lemma:deterministic}
  If a Verilog program $V$ admits behaviours $B_1$ and $B_2$, then $B_1$ and
  $B_2$ must be the same.

  \begin{equation*}
    \forall V, B_{1}, B_{2},\quad V \Downarrow B_{1} \land V \Downarrow B_{2} \implies B_{1} = B_{2}.
  \end{equation*}
\end{lemma}

\begin{proof}[Proof sketch]
  The Verilog semantics is deterministic because the order of operation of all
  the constructs is defined, so there is only one way to evaluate the module,
  and hence only one possible behaviour. This was proven for the small-step
  semantics shown in Fig.~\ref{fig:inference_module}.
\end{proof}

%%% Local Variables:
%%% mode: latex
%%% TeX-master: "../thesis"
%%% TeX-engine: luatex
%%% End:
