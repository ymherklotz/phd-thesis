\chapter{Introduction to Vericert}%
\label{sec:introduction-to-vericert}

\begin{chapsummary}
  This chapter describes the main architecture of the HLS tool, and the way in
  which the Verilog back end was added to \compcert{}.  This chapter also covers
  an example of converting a simple C program into hardware, expressed in the
  Verilog language.
\end{chapsummary}

\noindent Our solution to the verification problem in \gls{HLS} is Vericert, a
formally verified \gls{HLS} tool.  First, we will describe the main design
decisions behind Vericert in \cref{sec:itv:main-design-decisions} and we will
then give an example of a translation from C code into Verilog through Vericert
in \cref{sec:itv:translating-c-to-verilog}.

\section{Main Design Decisions}%
\label{sec:itv:main-design-decisions}

\paragraph{Choice of source language}

C was chosen as the source language as it remains the most common source
language amongst production-quality HLS tools~\cite{canis13_l,
  amd23_vitis_high_synth, intel20_hsc, pilato13_bambu}. This, in turn, may be
because it is \textquote[{\textcite{gajski10_what_hls}}]{[t]he starting point
  for the vast majority of algorithms to be implemented in hardware}, lending a
degree of practicality.  The availability of
\compcert{}~\cite{leroy09_formal_verif_realis_compil} also provides a solid
basis for formally verified C compilation.  Since a lot of existing code for HLS
is written in C, supporting C as an input language, rather than a custom
domain-specific language, means that \vericert{} is more practical.  We
considered Bluespec~\cite{nikhil04_bsv}, but decided that although it
\textquote[{\textcite{greaves19_resear_note}}]{can be classed as a high-level
  language}, it is too hardware-oriented to be suitable for traditional HLS.  We
also considered using a language with built-in parallel constructs that map well
to parallel hardware, such as occam~\cite{page91_compil_occam},
Spatial~\cite{koeplinger18_s} or Scala~\cite{bachrach12_chisel}.

\paragraph{Choice of target language}

Verilog~\cite{06_ieee_stand_veril_hardw_descr_languag} is an \gls{HDL} that can
be synthesised into logic cells which can either be placed onto an \gls{FPGA} or
turned into an \gls{ASIC}.  Verilog was chosen as the output language for
\vericert{} because it is one of the most popular HDLs and there already exist a
few formal semantics for it that could be used as a
target~\cite{lööw19_verif_compil_verif_proces, meredith10_veril}.  Bluespec,
previously ruled out as a source language, is another possible target and there
exists a formally verified translation to circuits of variants of Bluespec using
K\^{o}ika~\cite{bourgeat20_essen_blues} or
Fe-Si~\cite{braibant13_formal_verif_hardw_synth}.  However, Bluespec is mainly
targeted at being a source language for hardware design and would present
similar challenges to targeting a language like Verilog from a software
language.

% \JP{This needs an extra comment maybe?}\YH{Maybe about bluespec not being an
% ideal target language because it's quite
% high-level?} % but targeting this language would not be trivial as it is not meant to be targeted by an automatic tool, instead strives to a formally verified high-level hardware description language instead.

%\JW{Can we mention one or two alternatives that we considered? Bluespec or Chisel or one of Adam Chlipala's languages, perhaps?}

\paragraph{Choice of implementation language}

We chose Coq as the implementation language because of its mature support for
code extraction; that is, its ability to generate OCaml programs directly from
the definitions used in the theorems.  We note that other authors have had some
success reasoning about the HLS process using other theorem provers such as
Isabelle~\cite{ellis08_csicgfu}.
\compcert{}~\cite{leroy09_formal_verif_realis_compil} was chosen as the front
end because it has a well established framework for simulation proofs about
intermediate languages, and it already provides a validated C
parser~\cite{jourdan12_valid_lr_parser}.  The Vellvm
framework~\cite{zhao12_formal_llvm_inter_repres_verif_progr_trans} was also
considered because several existing HLS tools are already LLVM-based, but
additional work would be required to support a high-level language like C as
input.  The .NET framework has been used as a basis for other HLS tools, such as
Kiwi~\cite{greaves08_kiwi}, and LLHD~\cite{schuiki20_llhd} has been recently
proposed as an intermediate language for hardware design, but neither are
suitable for us because they lack formal semantics.

\begin{figure}
  \centering
  \begin{tikzpicture}
    [font=\strut\sffamily]
    \node[ir] (compcertc) {CompCert C};
    \node[right=2cm of compcertc] (frontend) {};
    \begin{pgfonlayer}{background}
          \node[bgbox,fill=bgbox1,minimum height=4cm,minimum width=3cm,yshift=3mm]
    (frontendbg) at (frontend) {};
    \end{pgfonlayer}
    \node[pass] (frontendpass) at (frontend) {$\vphantom{A}\cdots$};
    \node[right=3.5cm of frontend] (rtlghost) {};
    \begin{pgfonlayer}{background}
      \node[bgbox,fill=bgbox2,minimum height=4cm,minimum width=3cm,yshift=3mm]
      (rtlghostbg) at (rtlghost) {};
    \end{pgfonlayer}
    \node[ir,below=2mm of rtlghost,xshift=-8mm] (rtl) {\rtl{}};
    \node[pass,right=5mm of rtl] (rtlpass) {$\vphantom{A}\cdots$};
    \node[pass,above=2mm of rtlghost] (opts) {optimisations};
    \node[anchor=south,above=2mm of opts] (backend) {Back End};
    \node[ir,right=2cm of rtlghost] (nativecode) {Native Code};
    \path (backend) -| node {Front End} (frontend);
    \draw[ed] (compcertc) -- (frontendpass) -- ++(2cm,0) |- (rtl);
    \draw[ed] (rtl) -- (rtlpass) -- ++(1cm,0) |- (nativecode);
    \draw[very thick] (opts) to [out=310,in=20] (rtl);
    \draw[edr] (rtl) to [out=90,in=220] (opts);
    \node[pass,below=2.3cm of compcertc,xshift=1cm] (findbb) {find\\basic blocks};
    \node[ir,right=of findbb] (rtlblock) {\rtlblock{}};
    \draw[ed] (rtlblock) to [out=300,in=240,loop,looseness=10] (rtlblock);
    \node[pass,right=of rtlblock] (schedule) {schedule};
    \node[pass,below=of rtlblock] (ifconv) {if-conversion};
    \node[ir,right=of schedule] (rtlpar) {\rtlpar{}};
    \node[pass,below=of findbb,yshift=-2cm,xshift=-1cm] (hyperblock destruction) {hyperblock\\destruction};
    \node[ir,right=of hyperblock destruction] (rtlsubpar) {\rtlsubpar{}};
    \node[pass,right=of rtlsubpar] (htl generation) {\htl{} \\
      generation};
    \node[ir,right=of htl generation] (htl) {\htl{}};
    \node[pass,right=of htl] (bram insertion) {BRAM \\ insertion};
    \node[ir,below=of bram insertion] (htlmem) {\htl{}};
    \node[pass,left=of htlmem] (forward substitution) {forward \\ substitution};
    \node[ir,left=of forward substitution] (htlsubst) {\htl{}};
    \node[pass,left=of htlsubst] (verilog generation) {verilog \\ generation};
    \node[ir,left=of verilog generation] (verilog) {Verilog};

    \begin{pgfonlayer}{background}
      \node[bgbox,fit={(findbb)(rtlblock)(ifconv)(rtlpar)}] (schedulebox) {};
      \node[bgbox,fit={(hyperblock destruction)(bram
        insertion)(htlmem)(forward substitution)(verilog)}] (hardwaregenbox) {};
    \end{pgfonlayer}

      \path (hardwaregenbox.north west) --
      node[font=\sffamily\small,align=center,rotate=90,yshift=5mm] (hardwaregen label)
      {Hardware Generation\\\Cref{sec:hardware-generation}}
      (hardwaregenbox.south west);
      \path (schedulebox.north west) --
      node[font=\sffamily\small,align=center,rotate=90,yshift=14.5mm] (schedule label)
      {Hyperblock Scheduling\\\Cref{sec:hyperblock-scheduling}}
      (schedulebox.south west);

      \begin{pgfonlayer}{background}
        \node[bgbox,fill=bgbox4,fit={(hyperblock destruction)(bram
          insertion)(htlmem)(forward substitution)(verilog)(hardwaregen label)}]
        (hardwaregenboxb) {};
      \node[bgbox,fill=bgbox3,fit={(findbb)(rtlblock)(ifconv)(rtlpar)(schedule
        label)}] (scheduleboxb) {};
      \filldraw[bgbox,fill=bgbox3] (scheduleboxb.north west) rectangle
      (hardwaregenboxb.north east
      |- scheduleboxb.south east);
      \end{pgfonlayer}

    \path (rtlghostbg.west) -| node[font=\sffamily\small,anchor=south,rotate=90] {CompCert} (hardwaregen label);

    \draw[ed] (rtl) -- ++(0,-1.3cm) -| (findbb) -- (rtlblock);
    \draw[ed] (rtlblock) -- (schedule) -- (rtlpar);
    \draw[ed] (rtlpar) -- ++(0,-2.7cm) -| (hyperblock destruction) -- (rtlsubpar);
    \draw[ed] (rtlsubpar) -- (htl generation) -- (htl);
    \draw[ed] (htl) -- (bram insertion) -- (htlmem);
    \draw[ed] (htlmem) -- (forward substitution) -- (htlsubst);
    \draw[ed] (htlsubst) -- (verilog generation) -- (verilog);

    \node[blacknum] at (findbb.north west) {1};
    \node[blacknum] at (ifconv.north west) {2};
    \node[blacknum] at (schedule.north west) {3};
    \node[blacknum] at (hyperblock destruction.north west) {4};
    \node[blacknum] at (htl generation.north west) {5};
    \node[blacknum] at (bram insertion.north west) {6};
    \node[blacknum] at (forward substitution.north west) {7};
    \node[blacknum] at (verilog generation.north west) {8};
  \end{tikzpicture}
  \caption{\vericert{} as a Verilog back end to \compcert{}.}%
  \label{fig:rtlbranch}
\end{figure}

\paragraph{Architecture of \vericert{}}

An overview of \vericert{}'s workflow is given in \cref{fig:rtlbranch}, which
shows that \vericert{} branches off from \compcert{} at the \rtl{} stage,
followed by a number of transformations related to the scheduling instructions,
and finally transformations that generate the final hardware.

We select CompCert's register transfer language (\rtl{}) as the starting
point. Branching off \emph{before} this point (at CminorSel or earlier) denies
\compcert{} the opportunity to perform optimisations such as constant
propagation and dead-code elimination, which, despite being designed for
software compilers, have been found useful in HLS tools as
well~\cite{cong11_high_level_synth_fpgas}. And if we branch off \emph{after}
this point (at LTL or later) then \compcert{} has already performed register
allocation to reduce the number of registers and spill some variables to the
stack; this transformation is not required in HLS because there are many more
registers available, and these should be used instead of \gls{BRAM} whenever
possible.

\rtl{} is also attractive because it is the closest intermediate language to
LLVM \gls{IR}, which is used by several existing HLS compilers.  It has an
unlimited number of pseudo-registers, and is represented as a \gls{CFG} where
each instruction is a node with links to the instructions that can follow it.
\rtl{} does not have the SSA property, however, this is not required for the
translation to hardware and mainly assists the static analysis passes.  One
difference between LLVM \gls{IR} and \rtl{} is that \rtl{} includes operations
that are specific to the chosen target architecture; we chose to target the
x86\_32 back end because it generally produces relatively dense \rtl{} thanks to
the availability of complex addressing modes.  The translation from \rtl{} is
then performed as follows.

\begin{enumerate}[label=\protect\blacknum{\arabic*}]
\item Basic block generation creates basic blocks from the pure \rtl{}
  \gls{CFG}.
\item If-conversion combine basic blocks into a single hyperblock based on
  heuristics on the \gls{CFG} layout.  The hyperblock is represented as a basic
  block of predicated instructions with an exit instruction to leave the basic
  block prematurely.
\item The scheduler itself is written in unverified OCaml, and works similarly
  to those in existing HLS tools~\cite[]{canis13_l}: it takes a set of
  scheduling constraints that capture the clock period, available hardware
  resources, and dependencies between operations, encodes them all as a system
  of difference constraints (SDC)~\cite[]{cong06_sdc}, and then hands them off
  to a linear program solver.
\item Next, the hyperblocks are destroyed again to explicitly place instructions
  into individual states.
\item \htl{} is generated, which is a language that models an \gls{FSMD}.
\item \gls{BRAM} insertion, which generates a proper memory interface for any
  interaction with the stack.
\item To ensure that these assignments are actually performed in parallel, we
  have a final pass that performs \emph{forward
    substitution}~\cite[p.~109]{hopwood78_decom} to turn the sequence of Verilog
  blocking assignments into a sequence of nonblocking assignments. For example:
%
\begin{center}
\begin{tikzpicture}[>=Latex,shorten >=1pt,label/.style={circle,draw,fill=white,inner sep=0.4mm,font=\footnotesize}, bb/.style={align=left, draw=white, fill=black!5},font=\strut\sffamily]
\node[bb,inner sep=3mm] (initial) {\begin{minipage}{1.7cm}
\begin{minted}{systemverilog}
a = b * c;
d = a + d;
\end{minted}
\end{minipage}};
\node[right=5cm of initial,bb,inner sep=3mm] (scheduled) {%
  \begin{minipage}{3cm}
\begin{minted}{systemverilog}
a <= b * c;
d <= (b * c) + d;
\end{minted}
  \end{minipage}%
};
\draw[ed] (initial) -- node [pass] {forward\\substitution}
(scheduled);
\end{tikzpicture}
\end{center}
%
The two versions are semantically equivalent, but we find that the second, in
which both right-hand sides must be evaluated before either assignment is
performed, makes the downstream logic synthesis tools more likely to produce the
hardware we intend (which, in this particular example, involves exploiting a
fused multiply--accumulator unit if available).
\item Finally, syntactic Verilog is generated from \htl{}, which consists of
  translating the \gls{FSMD} into a case statement and implementing the memory
  interface in Verilog.
\end{enumerate}

\section{Translating C to Verilog by Example}%
\label{sec:itv:translating-c-to-verilog}

\Cref{fig:accumulator_c_rtl} illustrates the translation of a simple program
that stores and retrieves values from an array.  In this section, we describe
the stages of the \vericert{} translation, referring to this program as an
example.

\subsection{Translating C to \rtl{}}

The first stage of the translation uses unmodified \compcert{} to transform the
C input, shown in \cref{fig:accumulator_c}, into a \rtl{} intermediate
representation, shown in \cref{fig:accumulator_rtl}.  As part of this
translation, function inlining is performed on all functions, which allows us to
support function calls without having to support the \texttt{Icall} \rtl{}
instruction.  Although the duplication of the function bodies caused by inlining
can increase the area of the hardware, it can have a positive effect on latency
and is therefore a common HLS optimisation~\cite{noronha17_rapid_fpga}.
Scheduling in particular benefits from inlining of function calls so that the
instructions can be scheduled together and larger hyperblocks can be formed.
Inlining precludes support for recursive function calls, but this feature is not
supported in most HLS tools anyway~\cite{thomas16_srcht}.

\begin{figure}
  \centering
    \begin{subfigure}[b]{0.48\linewidth}
\begin{minted}[fontsize=\footnotesize,linenos,xleftmargin=20pt]{c}
int main() {
    int x[2] = {3, 6};
    int i = 1;
    return x[i];
}
\end{minted}
      \caption{Example C code passed to \vericert{}.}\label{fig:accumulator_c}
    \end{subfigure}\hfill%
    \begin{subfigure}[b]{0.48\linewidth}
\begin{minted}[fontsize=\footnotesize,linenos,xleftmargin=20pt]{c}
main() {
    9:  x5 = 3
    8:  int32[stack(0)] = x5
    7:  x4 = 6
    6:  int32[stack(4)] = x4
    5:  x1 = 1
    4:  x3 = stack(0) (int)
    3:  x2 = int32[x3 + x1 * 4 + 0]
        goto 1
    2:  x2 = 0
    1:  return x2
}
\end{minted}
      \caption{\rtl{} produced by the \compcert{} front end without any optimisations.}\label{fig:accumulator_rtl}
    \end{subfigure}
    \caption{Translating a simple program from C to \rtl{}.}\label{fig:accumulator_c_rtl}
\end{figure}

\subsection{Scheduling \rtl{} instructions}

The first step in the translation performed by Vericert is to schedule the
instructions according to the resource constraints imposed by the hardware
target.  An example of such a constraint is that two memory operations cannot be
performed in the same cycle.  The goal is to schedule as many instructions as
possible together to give the scheduler the most flexibility.  Vericert
therefore constructs hyperblocks from the \rtl{} \gls{CFG}, by building basic
blocks first and then performing if-conversion on blocks that should be merged.
This is shown in \cref{fig:accumulator_seq}, where all the instructions can fit
into a single basic block.  Each hyperblock can then be scheduled individually
using \gls{SDC} scheduling by specifying any necessary constraints, the result
of which is shown in \cref{fig:accumulator_par}, where one can see that
instructions have been separated into separate states and instructions that can
execute in parallel are placed into the same state.  Each memory operation is
placed into its own state.  The instructions are placed into additional bundles,
designated by parentheses, which could contain additional operations that are
chained sequentially within one clock cycle.

The \gls{SDC} scheduling algorithm itself is unverified but a verified validator
checks the resulting schedule against the unscheduled program.  This makes it
possible to change heuristics in the scheduler simply without it affecting the
proof.  The scheduled program language is called \rtlpar{} and specifies in
which cycle each instruction will be executed.  \rtlpar{} however still contains
the coarse-grained structure of hyperblocks, so this structure is destroyed to
produce \rtlsubpar{}, shown in \cref{fig:accumulator_par}.  The representation
is still hardware agnostic even if the current implementation of the scheduler
is specific to an \gls{FPGA} target, and therefore supports all \rtl{}
instructions.  Details about the scheduler and proof of correctness is given in
\cref{sec:hyperblock-scheduling}.  This representation is then ready to be
translated into hardware.

\begin{figure}
  \centering
    \begin{subfigure}[b]{0.48\linewidth}
\begin{minted}[fontsize=\footnotesize,linenos,xleftmargin=20pt]{c}
main() {
  8: {
    x5 = 3
    int32[stack(0)] = x5
    x4 = 6
    int32[stack(4)] = x4
    x1 = 1
    x3 = stack(0) (int)
    x2 = int32[x3 + x1 * 4 + 0]
    nop
    return x2
  }
}
\end{minted}
      \caption{Code in \rtlblock{} after basic blocks have been generated.}\label{fig:accumulator_seq}
    \end{subfigure}\hfill%
    \begin{subfigure}[b]{0.48\linewidth}
\begin{minted}[fontsize=\footnotesize,linenos,xleftmargin=20pt]{c}
main() {
  8: {
    (x5 = 3)
    (x4 = 6)
    (x1 = 1)
    (x3 = stack(0) (int))
  }
  14: { (int32[stack(0)] = x5) }
  15: { (int32[stack(4)] = x4) }
  16: { (x2 = int32[x3 + x1 * 4 + 0]) }
  17: { (return x2) }
}
\end{minted}
      \caption{\rtlsubpar{} code produced after scheduling and hyperblock destruction.}\label{fig:accumulator_par}
    \end{subfigure}
    \caption{Translating a simple program from C to \rtl{}.}\label{fig:accumulator_gblseqpar}
\end{figure}

% + TODO Explain the main mapping in a short simple way

% + TODO Clarify connection between CFG and FSMD

% + TODO Explain how memory is mapped \JW{I feel like this could use some sort
% of citation, but I'm not sure what. I guess this is all from "Hardware Design
% 101", right?}\YH{I think I found a good one actually, which goes over the
% basics.}  \JW{I think it would be worth having a sentence to explain how the C
% model of memory is translated to a hardware-centric model of memory. For
% instance, in C we have global variables/arrays, stack-allocated
% variables/arrays, and heap-allocated variables/arrays (anything else?). In
% Verilog we have registers and \gls{BRAM} blocks. So what's the correspondence
% between the two worlds?  Globals and heap-allocated are not handled,
% stack-allocated variables become registers, and stack-allocated arrays become
% \gls{BRAM} blocks? Am I close?}\YH{Stack allocated variables become \gls{BRAM}
% as well, so that we can deal with addresses easily and take addresses of any
% variable.} \JW{I see, thanks. So, in short, the only registers in your
% hardware designs are those that store things like the current state, etc. You
% generate a fixed number of registers every time you synthesis -- you don't
% generate extra registers to store any of the program variables. Right?}

% \JP{I've become less comfortable with this term, but it's personal preference
% so feel free to ignore. I think `generalised finite state machine' (i.e.\
% thinking of the entire `data-path' as contributing to the overall state) is
% more accurate.}\YH{Hmm, yes, I mainly chose FSMD because there is quite a lot
% of literature around it.  I think for now I'll keep it but for the final draft
% we could maybe change it.}  This means that the state transitions can be
% translated into a simple finite state machine (FSM) where each state contains
% data operations that update the memory and registers.

\subsection{Translating \rtlpar{} to \htl{}}\label{sec:itv:rtlpar-to-htl}

The next translation is from \rtlpar{}, the program formed of scheduled
hyperblocks, to an intermediate hardware translation language (\htl{}).  This
translation involves going from a \gls{CFG} representation of the computation to
an \gls{FSMD} representation~\cite{hwang99_ffplp}. An \gls{FSMD} is a
generalised state machine where the state is supplemented by registers and
memory that can store computations.  \Cref{fig:accumulator_diagram} shows the
resulting \gls{FSMD} architecture for the running example. The right-hand block
is the control logic that computes the next state, while the left-hand block
updates all the registers and \gls{BRAM} based on the current program state.
However, in general the state machine cannot be separated from the data path
completely, because state updates may depend on computations in the data path
and may depend on the value of registers.  Hence, an \htl{} program consists a
map from states to Verilog statements, which update both the state register as
well as other registers and interact with memory.

The \htl{} language was mainly introduced to simplify the proof of translation
from \rtl{} to Verilog, as these languages have very different semantics.  It
serves as an intermediate language with similar semantics to \rtl{} at the top
level, using maps to represents what to execute at every state, and similar
semantics to Verilog at the lower level by already using Verilog statements
instead of more abstract instructions to perform computations.  The next state
is also computed explicitly in each state by modifying the state register.XS
Compared to plain Verilog and due to using maps to represent the Verilog
statement that should execute at every state, \htl{} is simpler to manipulate
and analyse, thereby making it easier to prove optimisations like \gls{BRAM}
insertion.

\begin{figure}
  \centering
\definecolor{control}{HTML}{b3e2cd}
\definecolor{data}{HTML}{fdcdac}
\begin{tikzpicture}
  \begin{scope}[scale=1.3]
  \fill[control,fill opacity=1] (6.5,0) rectangle (12,5);
  \fill[data,fill opacity=1] (0,0) rectangle (5.5,5);
  \shade[left color=data, right color=control] (6.7,0) rectangle (5.3,5);
  \node[anchor=north west,font=\bfseries\sffamily] at (0.2,4.9) {Data Path};
  \node[anchor=north west,font=\bfseries\sffamily] at (6.9,4.9) {Control Logic};

  \filldraw[fill=white,rounded corners=3pt] (7,0.5) rectangle (11.5,2.2);
  \node at (8.2,2) {\footnotesize \texttt{Next State FSM}};
  \begin{scope}[xshift=5mm,yshift=-1mm]
  \foreach \x in {8,...,3}
    {\pgfmathtruncatemacro{\y}{8-\x}%
      \node[draw,circle,inner sep=0,minimum size=10,scale=0.8] (s\x) at
      (7.5+\y/2,1.35) {};}
  \node[font=\tiny] at (s8) {8};
  \node[font=\tiny] at (s7) {14};
  \node[font=\tiny] at (s6) {15};
  \node[font=\tiny] at (s5) {16};
  \node[font=\tiny] at (s4) {18};
  \node[font=\tiny] (s1c) at (s3) {17};
  \node[draw,circle,inner sep=0,minimum size=13,scale=0.8] (s1) at (s1c) {};
  \draw[-{Latex[length=1mm,width=0.7mm]}] (s1) to [loop,looseness=5,out=45,in=-45] (s1);
  \foreach \x in {8,...,4}
    {\pgfmathtruncatemacro{\y}{\x-1}\draw[-{Latex[length=1mm,width=0.7mm]}] (s\x) -- (s\y);}
  \draw[-{Latex[length=1mm,width=0.7mm]}] (7.2,1.7) to [out=0,in=100] (s8);
  \end{scope}

  \node[draw,fill=white] (nextstate) at (9.25,3) {\tiny \texttt{current state}};
  \draw[-{Latex[length=1mm,width=0.7mm]}] let \p1 = (nextstate) in
    (11.5,1.25) -| (11.75,\y1) -- (nextstate);
  \draw let \p1 = (nextstate) in (nextstate) -- (6,\y1) |- (6,1.5);
  \node[scale=0.5,rotate=60] at (7.5,0.75) {\texttt{clk}};
  \node[scale=0.5,rotate=60] at (7.7,0.82) {\texttt{reset}};
  \draw[-{Latex[length=1mm,width=0.7mm]}] (7.65,-0.5) -- (7.65,0.5);
  \draw[-{Latex[length=1mm,width=0.7mm]}] (7.45,-0.5) -- (7.45,0.5);

  \filldraw[fill=white,rounded corners=3pt] (2,0.5) rectangle (5,3);
  \filldraw[fill=white] (0.25,0.5) rectangle (1.5,2.75);
  \node at (2.6,2.8) {\footnotesize \texttt{Update}};
  \node[align=center] at (0.875,2.55) {\footnotesize \texttt{\gls{BRAM}}};
  \node[scale=0.5] at (4.7,1.5) {\texttt{state}};
  \draw[-{Latex[length=1mm,width=0.7mm]}] (6,1.5) -- (5,1.5);
  \draw[-{Latex[length=1mm,width=0.7mm]}] (6,1.5) -- (7,1.5);
  \node[scale=0.5,rotate=60] at (4.1,0.9) {\texttt{finished}};
  \node[scale=0.5,rotate=60] at (3.9,0.95) {\texttt{return\_val}};
  \node[scale=0.5,rotate=60] at (2.5,0.75) {\texttt{clk}};
  \node[scale=0.5,rotate=60] at (2.7,0.82) {\texttt{reset}};

  \node[scale=0.5,right,inner sep=5pt] (ram1) at (2,2.1) {\texttt{u\_en}};
  \node[scale=0.5,right,inner sep=5pt] (ram2) at (2,1.9) {\texttt{wr\_en}};
  \node[scale=0.5,right,inner sep=5pt] (ram3) at (2,1.7) {\texttt{addr}};
  \node[scale=0.5,right,inner sep=5pt] (ram4) at (2,1.5) {\texttt{d\_in}};
  \node[scale=0.5,right,inner sep=5pt] (ram5) at (2,1.3) {\texttt{d\_out}};

  \node[scale=0.5,left,inner sep=5pt] (r1) at (1.5,2.1) {\texttt{u\_en}};
  \node[scale=0.5,left,inner sep=5pt] (r2) at (1.5,1.9) {\texttt{wr\_en}};
  \node[scale=0.5,left,inner sep=5pt] (r3) at (1.5,1.7) {\texttt{addr}};
  \node[scale=0.5,left,inner sep=5pt] (r4) at (1.5,1.5) {\texttt{d\_in}};
  \node[scale=0.5,left,inner sep=5pt] (r5) at (1.5,1.3) {\texttt{d\_out}};

  \draw[-{Latex[length=1mm,width=0.7mm]}] (ram1) -- (r1);
  \draw[-{Latex[length=1mm,width=0.7mm]}] (ram2) -- (r2);
  \draw[-{Latex[length=1mm,width=0.7mm]}] (ram3) -- (r3);
  \draw[-{Latex[length=1mm,width=0.7mm]}] (ram4) -- (r4);
  \draw[-{Latex[length=1mm,width=0.7mm]}] (r5) -- (ram5);

  \draw[-{Latex[length=1mm,width=0.7mm]}] (4,0.5) -- (4,-0.5);
  \draw[-{Latex[length=1mm,width=0.7mm]}] (3.75,0.5) -- (3.75,-0.5);
  \draw[-{Latex[length=1mm,width=0.7mm]}] (2.45,-0.5) -- (2.45,0.5);
  \draw[-{Latex[length=1mm,width=0.7mm]}] (2.65,-0.5) -- (2.65,0.5);

  \foreach \x in {0,...,1}
  {\draw (0.25,1-0.25*\x) -- (1.5,1-0.25*\x); \node at (0.875,0.88-0.25*\x) {\tiny \x};}

  %\node[scale=0.5] at (1.2,2.2) {\texttt{wr\_en}};
  %\node[scale=0.5] at (1.2,2) {\texttt{wr\_addr}};
  %\node[scale=0.5] at (1.2,1.8) {\texttt{wr\_data}};
  %\node[scale=0.5] at (1.2,1.4) {\texttt{r\_addr}};
  %\node[scale=0.5] at (1.2,1.2) {\texttt{r\_data}};
  %
  %\node[scale=0.5] at (2.3,2.2) {\texttt{wr\_en}};
  %\node[scale=0.5] at (2.3,2) {\texttt{wr\_addr}};
  %\node[scale=0.5] at (2.3,1.8) {\texttt{wr\_data}};
  %\node[scale=0.5] at (2.3,1.4) {\texttt{r\_addr}};
  %\node[scale=0.5] at (2.3,1.2) {\texttt{r\_data}};
  %
  %\draw[-{Latex[length=1mm,width=0.7mm]}] (2,2.2) -- (1.5,2.2);
  %\draw[-{Latex[length=1mm,width=0.7mm]}] (2,2) -- (1.5,2);
  %\draw[-{Latex[length=1mm,width=0.7mm]}] (2,1.8) -- (1.5,1.8);
  %\draw[-{Latex[length=1mm,width=0.7mm]}] (2,1.4) -- (1.5,1.4);
  %\draw[-{Latex[length=1mm,width=0.7mm]}] (1.5,1.2) -- (2,1.2);

  \filldraw[fill=white] (2.8,3.25) rectangle (4.2,4.75);
  \node at (3.5,4.55) {\footnotesize \texttt{Registers}};
  \draw[-{Latex[length=1mm,width=0.7mm]}] (2,2.4) -| (1.75,4) -- (2.8,4);
  \draw[-{Latex[length=1mm,width=0.7mm]}] (4.2,4) -- (5.25,4) |- (5,2.4);
  \draw[-{Latex[length=1mm,width=0.7mm]}] (5.25,2.4) -- (5.9,2.4) arc (180:0:.1)
  -- (6.2,2.4) |- (7,1.8);

  \node[scale=0.5] at (3.5,4.2) {\texttt{reg\_2}};
  \node[scale=0.5] at (3.5,4) {\texttt{reg\_4}};
  \node[scale=0.5] at (3.5,3.8) {\texttt{reg\_6}};
  \node[scale=0.5] at (3.5,3.6) {\texttt{reg\_8}};
  \node[scale=0.5] at (3.5,3.4) {\texttt{reg\_10}};
\end{scope}
\end{tikzpicture}
%  \alt{Diagram displaying the data-path and its internal modules, as well as the control logic and its state machine.}
\caption{The FSMD for the example shown in \cref{fig:accumulator_c_rtl}, split
  into a data path and control logic for the next state calculation.  The update
  block takes the current state, current values of all registers and at most one
  value stored in the \gls{BRAM}, and calculates a new value that can either be
  stored back in the \gls{BRAM} or in a
  register.}\label{fig:accumulator_diagram}
\end{figure}

\begin{figure}
  \centering
  \inputminted[fontsize=\footnotesize,linenos,xleftmargin=20pt]{systemverilog}{figures/3-introduction-to-vericert/translated-verilog.sv}
  \caption{Verilog implementation of \rtl{} code produced by CompCert produced
    by scheduling the code, instantiating a BRAM and translating to Verilog.}
  \label{fig:accumulator_v}
\end{figure}

%\JP{Does it? Verilog has neither physical registers nor RAMs, just language constructs which the synthesiser might implement with registers and RAMs. We should be clear whether we're talking about the HDL representation, or the synthesised result: in our case these can be very different since we don't target any specific architectural features of an FPGA fabric of ASIC process.}
\paragraph{Translating memory}

Typically, HLS-generated hardware consists of a sea of registers and RAMs.  This
memory view is very different from the C memory model, so we perform the
following translation from \compcert{}'s abstract memory model to a concrete
\gls{BRAM}.\@ Variables that do not have their address taken are kept in
registers, which correspond to the registers in \rtl{}.  All address-taken
variables, arrays, and structs are kept in \gls{BRAM}.  The stack of the main
function becomes an unpacked array of 32-bit integers representing the
\gls{BRAM} block.  Any loads and stores are temporarily translated to direct
accesses to this array, where each address has its offset removed and is divided
by four.  In a separate \htl{}-to-\htl{} conversion, these direct accesses are
then translated to proper loads and stores that use an interface to communicate
with the \gls{BRAM}, shown on lines 31--32, 35--36 and 39--40 of
\cref{fig:accumulator_v}.  This pass inserts a \gls{BRAM} block with the
interface around the unpacked array.  Without this interface and without the
\gls{BRAM} block, the synthesis tool processing the Verilog hardware description
would not identify the array as a \gls{BRAM}, and would instead implement it
using many registers.  This interface is shown on lines 10--18 in the Verilog
code in \cref{fig:accumulator_v}.  A high-level overview of the architecture and
of the \gls{BRAM} interface can be seen in \cref{fig:accumulator_diagram}.

\paragraph{Translating instructions}

Most \rtl{} instructions correspond to hardware constructs.  For example, line 2
which in \cref{fig:accumulator_rtl} shows a 32-bit register \texttt{x5} being
initialised to 3, after which the control flow moves execution to line 3. This
initialisation is also encoded in the Verilog generated from \htl{} at state 8
in the always-block implementing the \gls{FSMD}, shown at line 43 in
\cref{fig:accumulator_v}.  Simple operator instructions are translated in a
similar way.  For example, the add instruction is just translated to the
built-in add operator, similarly for the multiply operator.  All 32-bit
instructions can be translated in this way, but some special instructions
require extra care. One such instruction is the \texttt{Oshrximm} instruction,
which is discussed further in
\cref{sec:algorithm:optimisation:oshrximm}. Another is the \texttt{Oshldimm}
instruction, which is a left rotate instruction that has no Verilog equivalent
and therefore has to be implemented in terms of other operations and proven to
be equivalent.  The only 32-bit instructions that we do not translate are
case-statements (\texttt{Ijumptable}) and those instructions related to function
calls (\texttt{Icall}, \texttt{Ibuiltin}, and \texttt{Itailcall}), because we
enable inlining by default.

\subsection{Translating \htl{} to Verilog}

Finally, we have to translate the \htl{} code into proper
Verilog. % and prove that it behaves the same as the \rtl{} according to the Verilog semantics.
The challenge here is to translate our FSMD representation into a Verilog AST.
However, as all the instructions in \htl{} are already expressed as Verilog
statements, only the top-level data-path and control logic maps need to be
translated to valid Verilog case-statements.  We also require declarations for
all the variables in the program, as well as declarations of the inputs and
outputs to the module, so that the module can be used inside a larger hardware
design.  In addition to translating the maps of Verilog statements, an
always-block that will behave like the \gls{BRAM} also has to be created, which
is only modelled abstractly at the \htl{} level.  \Cref{fig:accumulator_v} shows
the final Verilog output that is generated for our example.

Although this translation seems quite straight\-forward, proving that this
translation is correct is complex.  All the implicit assumptions that were made
in \htl{} need to be translated explicitly to Verilog statements and it needs to
be shown that these explicit behaviours are equivalent to the assumptions made
in the \htl{} semantics.  One main example of this is proving that the
specification of the \gls{BRAM} in \htl{} does indeed behave in the same as its
Verilog implementation.  We discuss these proofs in upcoming sections.

% In general, the generated Verilog structure has similar to that of the \htl{}
% code.  The key difference is that the control and datapath maps become Verilog
% case-statements.  Other additions are the initialisation of all the variables
% in the code to the correct bitwidths and the declaration of the inputs and
% outputs to the module, so that the module can be used inside a larger hardware
% design.

%%% Local Variables:
%%% mode: latex
%%% TeX-master: "../thesis"
%%% TeX-engine: luatex
%%% End:
