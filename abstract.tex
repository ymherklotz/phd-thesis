\addchap{Abstract}

Latency, throughput, and energy efficiency are become increasingly important,
leading to more custom hardware accelerators being designed for numerous
applications instead of using less efficient general processors.  Alas,
designing these accelerators can be a tedious and error-prone process,
especially when using \glspl{HDL} such as Verilog or VHDL which operate at the
register transfer level where the hardware needs to be described manually.  As
the complexity of these hardware designs increases, designing hardware at this
level becomes less productive and checking that the hardware designs behave as
expected becomes increasingly difficult.  An attractive alternative is
\emph{\gls{HLS}}, where hardware designs are automatically compiled from
software written in a high-level language like C.  This way, hardware design can
benefit from mature software development tools while working on the general
functionality of the hardware design, and then use a modern \gls{HLS} tools such
as \legup{}~\cite{canis11_legup}, Vitis HLS~\cite{amd23_vitis_high_synth}, Intel
i++~\cite{intel_hls}, Stratus HLS~\cite{roane23_autom_hw_sw_co_desig} and Bambu
HLS~\cite{bambu_hls} to produce the hardware design at the register transfer
level.  These \gls{HLS} tools promise designs with comparable performance and
energy-efficiency to those hand-written in an \gls{HDL}~\cite{homsirikamol+14,
  silexicahlshdl, 7818341}.  This reduces the time needed to design new hardware
accelerators and as the design is performed at a higher level, this should also
make the process less error-prone.

%%% Local Variables:
%%% mode: latex
%%% TeX-master: "thesis"
%%% End:
