\addchap{Abstract}

Latency, throughput, and energy efficiency are becoming increasingly important,
leading to custom hardware accelerators being designed for numerous applications
instead of using less efficient general processors.  Alas, designing these
accelerators can be a tedious and error-prone process, especially when using
\glspl{HDL} such as Verilog or VHDL, which operate at the register transfer
level.

An attractive alternative is \emph{\gls{HLS}}, where hardware designs are
automatically compiled from software written in a high-level language like C.
This way, hardware designers can benefit from mature software development tools
while working on the functionality of the design.  \gls{HLS} tools promise
designs with comparable performance and energy-efficiency to those hand-written
in an \gls{HDL}, reducing the time needed to design new accelerators.  Being
able to reasoning about behaviour at a higher level should also make the process
less error-prone.  Unfortunately, \gls{HLS} tools have been found to be
unreliable; Vivado HLS produces incorrect designs in 1.2\% of randomly generated
C programs, undermining testing that was performed at the higher level of
abstraction.

I propose a formally verified \gls{HLS} tool called Vericert, providing a
computer-checked proof that ensures it only generates hardware designs that
behave the same as the input software program.  Vericert extends CompCert, an
established formally verified C Compiler, with a hardware back end.  One expects
a verified tool to produce significantly worse hardware than existing optimising
\gls{HLS} tools, as each transformation has to be proven correct.  Indeed, an
initial version of Vericert without optimisations was up to $8\times$ slower
than a state-of-the-art \gls{HLS} tool called Bambu with many optimisations
switched off.

However, by verifying hyperblock scheduling in Vericert, a transformation which
parallelises the instructions in large regions of code without loops, hardware
produced by Vericert becomes only around $1.6\times$ slower than Bambu without
optimisations and $3.6\times$ slower than hardware produced by optimised Bambu.
This is encouraging, showing that a verified \gls{HLS} tool can be compared with
an existing \gls{HLS} tool with some optimisations turned off, while being
guaranteed to generate correct hardware designs.

%%% Local Variables:
%%% mode: latex
%%% TeX-master: "thesis"
%%% TeX-engine: luatex
%%% End:
