\addchap{Abstract}

Latency, throughput, and energy efficiency are becoming increasingly important,
leading to custom hardware accelerators being designed for numerous applications
instead of using less efficient general processors.  Alas, designing these
accelerators can be a tedious and error-prone process, especially when using
\glspl{HDL} such as Verilog or VHDL, which operate at the register transfer
level where the hardware needs to be described manually.

An attractive alternative is \emph{\gls{HLS}}, where hardware designs are
automatically compiled from software written in a high-level language like C.
This way, hardware design can benefit from mature software development tools
while working on the general functionality of the hardware design, while promise
designs with comparable performance and energy-efficiency to those hand-written
in an \gls{HDL}.  This reduces the time needed to design new accelerators, but
it should also make the process less error-prone, as one can reason about the
behaviour of the hardware at a higher level.  Unfortunately, \gls{HLS} tools
have been found to be unreliable, for example, Vivado HLS produces incorrect
designs in up to 1.2\% of randomly generated C programs.  This undermines
testing that was performed at the higher level of abstraction.

We propose a formally verified \gls{HLS} tool called Vericert, which extends
CompCert, an existing formally verified C compiler, with a hardware back end.
Vericert branches off from the CompCert middle end and first parallelises the
instructions by scheduling hyperblocks, large regions of code without loops.
Scheduling places instructions into clock cycles, producing a finite-state
machine.  This state machine is then refined and transformed into a subset of
Verilog, which can subsequently be synthesised to a netlist.  Compared to a
state-of-the-art \gls{HLS} tool called Bambu, Vericert produces designs that are
around $1.6\times$ slower than Bambu with some optimisations turned off, and
$3.6\times$ slower than Bambu with the default optimisations turned on.

%%% Local Variables:
%%% mode: latex
%%% TeX-master: "thesis"
%%% TeX-engine: luatex
%%% End:
